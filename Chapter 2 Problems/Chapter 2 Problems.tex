\documentclass[12pt,letterpaper]{article}
\usepackage{preamble}
% \DeclarePairedDelimiter{\abs}{\lvert}{\rvert}

\newcommand\course{311 Self Study}
\newcommand\hwnumber{2}
\newcommand\userID{Rohit Rao}

\begin{document}
\begin{itemize}[leftmargin=!,labelindent=5pt]
    \item [2.2.1] Verify, using the definition of convergence of a sequence, that the following sequences converge to the proposed limit.
        \begin{enumerate}
            \item lim $\frac{1}{6n^2 + 1}=0$
                \begin{proof}
                    Let $\epsilon > 0$ be arbitrary.
                    Choose $N \in \mathbb{N}$ with $N > \frac{1}{\sqrt{6\epsilon}}$.
                    To verify let $n \geq N$ which means $\abs{\frac{1}{6n^2 + 1}} < \abs{\frac{1}{6\frac{1}{6\epsilon} + 1}} = \abs{\frac{1}{\frac{1}{\epsilon} + 1}} = \frac{\epsilon}{\epsilon + 1} < \epsilon$ as desired.
                \end{proof}
            \item lim $\frac{3n+1}{2n + 5}=\frac{3}{2}$
                \begin{proof}
                    Let $\epsilon > 0$ be arbitrary.
                    Choose $N \in \mathbb{N}$ with $N > \frac{13-10\epsilon}{4\epsilon}$.
                    To verify let $n \geq N$ and so $\abs{\frac{3n+1}{2n + 5}} < \abs{\frac{3(\frac{13-10\epsilon}{4\epsilon})+1}{2(\frac{13-10\epsilon}{4\epsilon}) + 5}}= \frac{3}{2}-\epsilon$.
                    So, $\abs{\frac{3n+1}{2n + 5}-\frac{3}{2}}< \epsilon$ as desired.
                \end{proof}
            \item lim $\frac{2}{\sqrt{n+3}}=0$
                \begin{proof}
                    Let $\epsilon > 0$ be arbitrary. Choose $N \in \mathbb{N}$ with $N > \frac{4}{\epsilon^2}-3$. To verify let $n \geq N$ which means $\abs{\frac{2}{\sqrt{n+3}}}<\abs{\frac{2}{\sqrt{(\frac{4}{\epsilon^2}-3)+3}}}=\epsilon$ as desired.
                \end{proof}
        \end{enumerate}
    \item [2.2.7] Informally speaking, the sequence $\sqrt{n}$ “converges to infinity.”
        \begin{itemize}
            \item [(a)] Imitate the logical structure of Definition 2.2.3 to create a rigorous definition for the mathematical statement lim $x_n = \infty$. Use this definition to prove lim $\sqrt{n} = \infty$. 
            
                A sequence $(a_n)$ “converges to infinity” if, for every positive number $\epsilon$, there exists an $N \in \mathbb{N}$ such that whenever $n \geq N$ it follows that $a_n > \epsilon$.
                
                \begin{proof}
                    Let $\epsilon > 0$ be arbitrary. Choose $N \in \mathbb{N}$ with $N > \epsilon^2$. To verify let $n \geq N$ which means $\sqrt{n} > \sqrt{\epsilon^2} = \epsilon$ as desired.
                \end{proof}
            \item [(b)] What does your definition in (a) say about the particular sequence $(1,0,2,0,3,0,4,0,5,0,...)$?
            
            It does not converge to infinity.
        \end{itemize}
    \item [2.2.8] Here are two useful definitions:
        \begin{enumerate}
            \item A sequence $(a_n)$ is \textit{eventually} in a set $A \subseteq \mathbb{R}$ if there exists an $N \in \mathbb{N}$ such that $a_n \in A$ for all $n \geq N$.
            \item A sequence $(a_n)$ is \textit{frequently} in a set $A \subseteq \mathbb{R}$ if for every $N \in \mathbb{N}$, there exists an $n \geq N$ such that $a_n \in A$.
        \end{enumerate}
        \begin{itemize}
            \item [(a)] Is the sequence $(-1)^n$ eventually or frequently in the set $\{1\}$?
            
            Frequently
            \item [(b)] Which definition is stronger? Does frequently imply eventually or does eventually imply frequently?
            
            Eventually is stronger; eventually implies frequently.
            \item [(c)] Give an alternate rephrasing of Definition 2.2.3B using either frequently or eventually. Which is the term we want?
            
            A sequence $(a_n)$ converges to $a$ if, given any $\epsilon$-neighborhood $V_\epsilon(a)$ of $a$, $(a_n)$ is eventually in the set $V_\epsilon(a)$.
            \item [(d)] Suppose an infinite number of terms of a sequence $(x_n)$ are equal to 2. Is $(x_n)$ necessarily eventually in the interval $(1.9, 2.1)$? Is it frequently in $(1.9, 2.1)$?
            
            $x_n$ is not necessarily eventually in the interval $(1.9, 2.1)$, but it is frequently in $(1.9, 2.1)$. 
            For example, $(2,2,2,2,...)$ is eventually in the interval $(1.9, 2.1)$, but $(2,-2,2,-2,...)$ is only frequently in the interval $(1.9, 2.1)$.
        \end{itemize}
    \item [2.3.3] \textbf{(Squeeze Theorem)}. Show that if $x_n \leq y_n \leq z_n$ for all $n \in \mathbb{N}$, and if lim $x_n =$ lim $z_n = l$, then lim $y_n = l$ as well.
        \begin{proof}
            Assume that $\forall n \in \mathbb{N}, x_n \leq y_n \leq z_n$ and $\lim x_n = \lim z_n = l$.
            Suppose $n \in \mathbb{N}$.
            Then, by the Order Limit Theorem, since $x_n \leq y_n$, $\lim y_n \geq \lim x_n = l$.
            Similarly, since $y_n \leq z_n$, $\lim y_n \leq \lim z_n = l$.
            So, $l \leq \lim y_n \leq l$.
            Thus, $\lim y_n = l$.
        \end{proof}
    \item [2.3.5] Let $(x_n)$ and $(y_n)$ be given, and define $(z_n)$ to be the “shuffled” sequence \\$(x_1, y_1, x_2, y_2, x_3, y_3, . . . , x_n, y_n, . . . )$. Prove that $(z_n)$ is convergent if and only if $(x_n)$ and $(y_n)$ are both convergent with lim $x_n =$ lim $y_n$.
        \begin{proof}
            \ \\
            First we prove that if $(z_n)$ is convergent then $(x_n)$ and $(y_n)$ are both convergent with $\lim x_n = \lim y_n$.
            Assume $(z_n)$ is convergent to $z$.
            Let $\epsilon > 0$ be arbitrary.
            Because $(z_n)$ is convergent to $z$ we know there exists an $N \in \bbN$ such that for all $n \geq N$, $\abs{z_n - z} < \epsilon$.
            So, $\epsilon > \abs{x_n-z}$ for all $n \geq N$ because $x_n = z_{2n-1}$ and $2n-1 \geq N$.
            Similarly, $\epsilon > \abs{y_n-z}$ for all $n \geq N$ because $y_n = z_{2n}$ and $2n \geq N$.
            Therefore, $\lim x_n = z = \lim y_n$.

            Next we prove that if $(x_n)$ and $(y_n)$ are both convergent with $\lim x_n = \lim y_n$ then $(z_n)$ is convergent.
            Assume $(x_n)$ and $(y_n)$ are both convergent with $\lim x_n = \lim y_n = l$.
            Let $\epsilon > 0$ be arbitrary.
            Then, we know that there exists an $N_1 \in \bbN$ such that for all $n \geq N_1$, $\abs{x_n - l} < \epsilon$.
            Similarly, we know that there exists an $N_2 \in \bbN$ such that for all $n \geq N_2$, $\abs{y_n - l} < \epsilon$.
            Let $N = $ max $\{2N_1, 2N_2\}$.
            Then, for all $n \geq N$, we have that $\abs{z_n - l} < \epsilon$ since $(z_n)$ consists of alternating elements from $(x_n)$ and $(y_n)$ which means after $N$ both $\abs{x_n - l} < \epsilon$ and $\abs{y_n - l} < \epsilon$.
            Thus, $(z_n)$ is convergent.
        \end{proof}
    \newpage
    \item [2.3.10] If $(a_n) \to 0$ and $\abs{b_n - b} \leq a_n$, then show that $(b_n) \to b$.
        \begin{proof}
            Assume $(a_n) \to 0$ and $\abs{b_n - b} \leq a_n$. 
            Let $\epsilon > 0$ be arbitrary.
            Since $(a_n) \to 0$, we know that there exists $N \in \bbN$ such that for all $n \geq N, \abs{a_n} < \epsilon$.
            Then, for all $n \geq N$, $\abs{b_n - b} \leq a_n < \epsilon$ which means $\abs{b_n - b} < \epsilon$, and so $(b_n) \to b$.
        \end{proof}
    \item [2.4.1] Complete the proof of Theorem 2.4.6 by showing that if the series $\sum_{n=0}^{\infty}2^nb_{2^n}$ diverges, then so does $\sum_{n=1}^{\infty}b_n$. Example 2.4.5 may be a useful reference.
        \begin{proof}
            Assume the series $\sum_{n=0}^{\infty}2^nb_{2^n}$ diverges.
            Then the monotone sequence of its partial sums $(t_k) = b_1 + 2b_2 + ... + 2^kb_{2^k}$ must be unbounded.
            We need to show that $\sum_{n=1}^{\infty}b_n$ diverges, namely that $(s_m) = b_1 + b_2 + ... + b_m$ is unbounded.
            Fix $m$ and choose an arbitrary $k$.
            Then,
            \begin{align*}
                s_{2^k} &= b_1 + b_2 + (b_3 + b_4) + (b_5 + b_6 + b_7 + b_8) + ... + (b_{2^{k-1}+1} + ... + b_{2^k})\\
                &\geq b_1 + b_2 + (b_4 + b_4) + (b_8 + b_8 + b_8 + b_8) + ... + (b_{2^k} + ... + b_{2^k})\\
                &= b_1 + b_2 + 2b_4 + 4b_8 + ... + 2^{k-1}b_{2^k}\\
                &= \frac{1}{2} (2b_1 + 2b_2 + 4b_4 + 8b_8 + ... + 2^kb_{2^k})\\
                &= \frac{1}{2} (b_1 + t_k).
            \end{align*}
            Since $(t_k)$ is unbounded and $s_m \geq \frac{t_k}{2}$, it must mean that $(s_m)$ is unbounded, and so $\sum_{n=1}^{\infty}b_n$ diverges.
        \end{proof}
    \item [2.4.2]
        \begin{itemize}
            \item [(a)] Prove that the sequence defined by $x_1 = 3$ and $x_{n+1} = \frac{1}{4-x_n}$ converges.
                \begin{proof}
                    We need to show that the sequence is monotone and bounded, then we can apply the Monotone Convergence Theorem.
                    First, we need to use induction to show that the sequence is monotone. 
                    
                    Base Case ($n=1$): $x_{1+1} = x_2 = \frac{1}{4-x_1} = \frac{1}{4-3} = 1$. $3 > 1$ so $x_1 > x_2$.

                    Induction Step:
                    \begin{itemize}
                        \item Suppose $k \in \bbN$ such that $k \geq 2$.
                        \item Assume for all natural numbers $i < k$, $x_i > x_{i+1}$.
                        \item Need to show that $x_{k} > x_{k+1}$.
                            $x_{k} = \frac{1}{4-x_{k-1}}$ and $x_{k+1} = \frac{1}{4-x_k}$.
                            By the induction hypothesis, we know that $x_{k-1} > x_{k}$.
                            So, $4 - x_{k-1} < 4 - x_k$ which means $\frac{1}{4-x_{k-1}} > \frac{1}{4-x_k}$.
                            Thus, $x_{k} > x_{k+1}$.
                    \end{itemize}
                    By the principle of complete induction, the sequence is decreasing.
                    Now, it is clear that $x_1 = 3$ is the upper bound for the sequence.
                    Next, we need to use induction to show that the sequence is bounded below by $0$.

                    Base Case ($n=1$): $x_{1+1} = x_2 = \frac{1}{4-x_1} = \frac{1}{4-3} = 1 > 0$.
                    
                    Induction Step:
                    \begin{itemize}
                        \item Suppose $k \in \bbN$ such that $k \geq 2$.
                        \item Assume for all natural numbers $i < k$, $x_i > 0$.
                        \item Need to show that $x_{k} > 0$.
                            $x_{k} = \frac{1}{4-x_{k-1}}$.
                            By the induction hypothesis, we know that $x_{k-1} > 0$.
                            Since, $3 \geq x_{k-1} > 0$, $\frac{1}{4-x_{k-1}} > 0$.
                            Thus, $x_{k} > 0$.
                    \end{itemize}
                    By the principle of complete induction, the sequence is always greater than $0$.
                    So, the sequence is bounded below by $0$.
                    Thus, the sequence is monotone and bounded. So, by the Monotone Convergence Theorem, the sequence converges.
                \end{proof}
            \item [(b)] Now that we know lim $x_n$ exists, explain why lim $x_{n+1}$ must also exist and equal the same value.
                
                Since lim $x_n$ exists, lim $x_{n+1}$ must also exist and be the same value because these sequences only differ by their starting value and an index of 1.
                This means that their end behavior (where the limit will end up) will remain identical.
            \item [(c)] Take the limit of each side of the recursive equation in part (a) of this exercise to explicitly compute lim $x_n$.
            
                Let $L = \lim x_n = \lim x_{n+1}$.
                Then, $L = \frac{1}{4-L}$ which becomes $L^2 - 4L + 1 = 0$ which results in $L = 2\pm \sqrt{3}$.
                Since the sequence is decreasing, the only answer is $L = 2 - \sqrt{3}$.
        \end{itemize}
    \item [2.4.6] \textbf{(Limit Superior)}. Let $(a_n)$ be a bounded sequence.
        \begin{itemize}
            \item [(a)] Prove that the sequence defined by $y_n = \sup\{a_k : k \geq n\}$ converges.
                \begin{proof}
                    First, we need to prove that $(y_n)$ is montonely decreasing, that is $y_{i+1} \leq y_i$ for all $i \in \bbN$.
                    Suppose $i \in \bbN$.
                    $y_{i+1} = \sup \{a_k : k \geq i+1\}$ and $y_{i} = \sup \{a_k : k \geq i\}$.
                    Since $\{a_k : k \geq i+1\} \subseteq \{a_k : k \geq i\}$, $\sup \{a_k : k \geq i+1\} \leq \sup \{a_k : k \geq i\}$.
                    So, $y_{i+1} \leq y_i$.
                    Next, we need to prove that $(y_n)$ is bounded, that is it has a lower bound and upper bound.
                    Since $(a_n)$ is bounded, let $L$ be the lower bound of $(a_n)$.
                    Then, by the way $y_n$ is defined, all of its elements come from $(a_n)$ which means $L$ is still a lower bound for $(y_n)$.
                    Since $(y_n)$ is decreasing, it is also bounded from above, namely its first element is an upper bound.
                    So, $(y_n)$ is bounded.
                    Thus, $(y_n)$ converges by the Monotone Convergence Theorem.
                    \end{proof}

            \item [(b)] The limit superior of $(a_n)$, or $\lim \sup a_n$, is defined by $\lim \sup a_n = \lim y_n$, where $y_n$ is the sequence from part (a) of this exercise. Provide a reasonable definition for $\lim \inf a_n$ and briefly explain why it always exists for any bounded sequence.
                
                \ \ \ \ $\lim \inf a_n = z_n$ where $z_n = \inf\{a_k : k \geq n\}$.
                It always exists for any bounded sequence because of the Monotone Convergence Theorem.
                We can apply it here because we know $(z_n)$ is bounded which can be shown by applying similar logic to the proof above, namely that $(a_n)$ is bounded.
                We also know that $(z_n)$ is increasing by applying similar logic to the proof above, namely that $\{a_k : k \geq n+1\} \subseteq \{a_k : k \geq n\}$ where $n \in \bbN$ and since we are taking infinums of the smaller set, then it must hold true that $z_{n+1} \geq z_n$.
                So, since it is both bounded and monotone, we apply the Monotone Convergence Theorem to show it converges.
            \item [(c)] Prove that $\lim \inf a_n \leq \lim \sup a_n$ for every bounded sequence, and give an example of a sequence for which the inequality is strict.
                \begin{proof}
                    Suppose $n \in \mathbb{N}$. 
                    Let $y_n = \sup\{a_k : k \geq n\}$ and $z_n = \inf\{a_k : k \geq n\}$.
                    Then, $z_n \leq y_n$ because $z_n$ is the infimums and $y_n$ is the supremums.
                    So, by the Order Limit Theorem, $\lim z_n \leq \lim y_n$.
                    Thus $\lim \inf a_n \leq \lim \sup a_n$.
                \end{proof}
                
                Example: Let $(a_n) = (-1,1,-1,1,-1,1,...)$. Then, $\lim \inf a_n = -1$ and $\lim \sup a_n = 1$.
                Since $-1<1$, $\lim \inf a_n < \lim \sup a_n$ in this case.
            \item [(d)] Show that $\lim \inf a_n = \lim \sup a_n$ if and only if $\lim a_n$ exists. In this case, all three share the same value.
                \begin{proof}
                    \ \\
                    First, we prove that if $\lim \inf a_n = \lim \sup a_n$, then $\lim a_n$ exists.
                    Assume $\lim \inf a_n = \lim \sup a_n$. Let $\epsilon > 0$ be arbitrary.
                    Then, we know that there exists an $N \in \mathbb{N}$ such that for all $n \geq N$, $\abs{ \inf a_n - L } < \epsilon$ and $\abs{ \sup a_n - L } < \epsilon$.
                    Since $\inf a_n \leq a_n \leq \sup a_n$, it must be true that $\abs{a_n - L } < \epsilon$ for all $n \geq N$.
                    Thus, $\lim a_n$ exists.
                    \ \\

                    Next, we prove that if $\lim a_n$ exists, then $\lim \inf a_n = \lim \sup a_n$.
                    Assume $\lim a_n$ exists.
                    Let $\epsilon > 0$ be arbitrary.
                    Then, there exists an $N \in \mathbb{N}$ such that for all $n \geq N$, $\abs{a_n - L} < \epsilon$.
                    Since every element after $a_n$ must be within the bounds $L-\epsilon$ and $L + \epsilon$, it follows that $\inf a_n$ and $\sup a_n$ must exist within these bounds as well. 
                    Then, $L-\epsilon \leq \lim\inf a_n \leq L + \epsilon$ and $L-\epsilon \leq \lim\sup a_n \leq L + \epsilon$.
                    Thus, since $\epsilon$ is arbitrary, we get that $\lim\inf a_n = L = \lim\sup a_n$.
                \end{proof}
        \end{itemize}
    \item [2.5.2]
        \begin{itemize}
            \item [(a)] Prove that if an infinite series converges, then the associative property holds. Assume $a_1 + a_2 + a_3 + a_4 + a_5 +...$ converges to a limit $L$ (i.e., the sequence of partial sums $(s_n) \to L)$. Show that any regrouping of the terms $(a_1 +a_2 +...+a_{n_1})+(a_{n_1+1} +...+a_{n_2})+(a_{n_2+1} +...+a_{n_3})+...$ leads to a series that also converges to $L$.
                \begin{proof}
                    Assume $a_1 + a_2 + a_3 + a_4 + a_5 +...$ converges to a limit $L$, that is $s_n = \sum_{i=1}^{i=n} a_i$ and $\lim s_n = L$.
                    Suppose $b_n = \sum_{j=1}^{j=n} s_{n_j}$.
                    Then, by the way $s_n$ and $b_n$ are defined, we know that the sequence of partial sums of $b_n$ is a subsequence of the sequence of partial sums of $s_n$ (because the terms of $b_n$ are simply a different grouping of the terms in $s_n$).
                    So, since $(s_n)$ converges to $L$, $(b_n)$ must also converge to $L$ since it is a subsequence.
                    Thus, any regrouping of the terms $(a_1 +a_2 +...+ a_{n_1}) +(a_{n_1+1} +...+ a_{n_2}) + (a_{n_2+1} +...+a_{n_3})+...$ leads to a series that also converges to $L$.
                \end{proof}
            \item [(b)] Compare this result to the example discussed at the end of Section 2.1 where infinite addition was shown not to be associative. Why doesn’t our proof in (a) apply to this example?
            
                The series did not converge in that example.
        \end{itemize}

    \item [2.5.5] Extend the result proved in Example 2.5.3 to the case $\abs{b} < 1$. Show lim$(b^n) = 0$ whenever $-1 < b < 1$.
        \begin{proof}\ \\
            Consider the case $0<b<1$.
            Then, since $b > b^2 > b^3 > ... > 0$, the sequence $(b^n)$ is bounded and decreasing which means we can apply the Monotone Convergence Theorem to conclude that $(b^n)$ converges to $l$ such that $b>l\geq 0$.
            Since $(b^{2n})$ is a subsequence, $(b^{2n})$ converges to $l$.
            Then, $(b^{2n}) = (b^n)(b^n)$ which means it converges to $l*l = l^2$.
            Because limits are unique, $l^2 = l$ which means $l = 1$ or $l=0$.
            Since $0 \leq l < b < 1$, $l = 0$.

            Consider the case $b = 0$.
            Then, the sequence is $(0,0, 0, ...)$ which converges to $0$.
            
            Consider the case $-1<b<0$.
            Let $\epsilon>0$ be arbitrary.
            Since $0<\abs{b} < 1$, we know from the first case that there exists an $N \in \mathbb{N}$ such that for all $n \geq N$, $\abs{b^n - 0} < \epsilon$ so $\abs{b^n} < \epsilon$. 
            Because $\abs{b^n} = \abs{\abs{b}^n}$ for $n \geq N$, $\lim(b^n) = 0$ in this case as well. 
            
            Combining the above 3 cases, we get that lim$(b^n) = 0$ over the interval $-1 < b < 1$.
        \end{proof}
    \item [2.5.6] Let $(a_n)$ be a bounded sequence, and define the set $S=\{x \in \mathbb{R}:x<a_n$ for infinitely many terms $a_n\}$. Show that there exists a subsequence $(a_{n_k})$ converging to $s = \sup S$. (This is a direct proof of the Bolzano–Weierstrass Theorem using the Axiom of Completeness.)
        \begin{proof}
            Let $(a_n)$ be a bounded sequence, and define the set $S=\{x \in \mathbb{R}:x<a_n$ for infinitely many terms $a_n\}$.
            Then, since $(a_n)$ is a bounded sequence, $S$ must be nonempty and bounded from above.
            By the Axiom of Completeness, $S$ must have a supremum.
            Let $s = \sup S$.
            Suppose $\epsilon > 0$.
            Then, since $s - \epsilon \in S$ and $s - \epsilon < s$, we know by the way $S$ is defined that there exist infinitely many terms from $(a_n)$ such that $a_n > s - \epsilon$.
            So, we can construct the subsequence in the following manner:

            First, choose $k=1$ and pick $a_{n_1}$ to add to the subsequence such that $a_{n_1} \in (s - \frac{1}{1}, s]$.

            Next, choose $k>1$ and pick $a_{n_k}$ to add to the subsequence such that $a_{n_k} \in (s - \frac{1}{k}, s]$.

            Repeating the 2nd step while always choosing $k$ such that it is strictly greater than the previous choice will yield in the desired subsequence.
            This is because it is guaranteed there are always infinitely many terms left to pick from the sequence and add to the subsequence by the way $S$ is defined.
            Thus, this method creates a subsequence of $(a_n)$ which converges to $s$ as the $\frac{1}{k}$ term becomes smaller, that is $\abs{s - a_{n_k}} < \frac{1}{k}$.

        \end{proof}
    \item [2.6.2] Supply a proof for Theorem 2.6.2: Every convergent sequence is a Cauchy sequence
        \begin{proof}
            Assume the sequence $(x_n)$ converges to $x$.
            Let $\epsilon > 0$.
            Then, there exists an $N \in \bbN$ such that for all $m,n \geq N$, $\abs{x_n - x} < \frac{\epsilon}{2}$ and $\abs{x_m - x} < \frac{\epsilon}{2}$.
            Suppose $m, n \in \bbN$ such that $m \geq N$ and $n \geq N$.
            Then, by the triangle inequality:
            \begin{align*}
                \abs{x_n - x_m} &= \abs{x_n - x + x - x_m}\\
                &\leq \abs{x_n - x} + \abs{x_m - x}\\
                &< \frac{\epsilon}{2} + \frac{\epsilon}{2} = \epsilon.
            \end{align*}
            Thus, $\abs{x_n - x_m} < \epsilon$ which means $(x_n)$ is a cauchy sequence.
        \end{proof}
    \item [2.6.3]
        \begin{itemize}
            \item [(a)] Explain how the following pseudo-Cauchy property differs from the proper definition of a Cauchy sequence: A sequence $(s_n)$ is pseudo-Cauchy if, for all $\epsilon > 0$, there exists an $N$ such that if $n \geq N$, then $\abs{s_{n+1} - s_n} < \epsilon.$

                The pseudo-Cauchy property requires $N$ to be large enough so that summing two "consecutive" terms after the $N$'th term (inclusive) becomes less than $\epsilon$.
                The proper Cauchy property requires $N$ to be large enough so that summing "any" two terms after the $N$'th term (inclusive) becomes less than $\epsilon$.
            \item [(b)] If possible, give an example of a divergent sequence $(s_n)$ that is pseudo-Cauchy.
            
                The harmonic series $\sum_{n=1}^{\infty} \frac{1}{n}$ is divergent but still a pseudo-Cauchy sequence.
        \end{itemize}
    \item [2.6.4] Assume $(a_n)$ and $(b_n)$ are Cauchy sequences. Use a triangle inequality argument to prove $c_n = \abs{a_n - b_n}$ is Cauchy.
        \begin{proof}
            Suppose $\epsilon > 0$.
            Assume $(a_n)$ and $(b_n)$ are Cauchy sequences.
            Then, there exists an $N_1 \in \bbN$ and $N_2 \in \bbN$ such that for all $n_1, m_1 \geq N_1$, $\abs{a_{n_1} - a_{m_1}} < \frac{\epsilon}{2}$ and for all $n_2, m_2 \geq N_2$, $\abs{b_{n_2} - b_{m_2}} < \frac{\epsilon}{2}$.
            Let $N = \max \{N_1, N_2\}$.
            Then, for all $n, m \geq N$, $\abs{a_n - a_m} < \frac{\epsilon}{2}$ and $\abs{b_n - b_m} < \frac{\epsilon}{2}$.
            Now we need to show $(c_n)$ is cauchy. Suppose $n, m \geq N$. Then,
            \begin{align*}
                \abs{c_n - c_m} &= \abs{\abs{a_n - b_n} - \abs{a_m - b_m}}\\
                &\leq \abs{(a_n - b_n) - (a_m - b_m)}\\
                &= \abs{(a_n - a_m) + (b_n - b_m)}\\
                &\leq \abs{a_n - a_m} + \abs{b_n - b_m}\\
                &< \frac{\epsilon}{2} + \frac{\epsilon}{2} = \epsilon.
            \end{align*}
            Since $\abs{c_n - c_m} < \epsilon$, $(c_n)$ is a cauchy sequence.
        \end{proof}
    \newpage
    \item [2.7.4] Give an example to show that it is possible for both $\sum{x_n}$ and $\sum{y_n}$ to diverge but for $\sum{x_n y_n}$ to converge.
    
    One example would be the harmonic series $\sum_{n=1}^{\infty} \frac{1}{n}$.
    If $x_n = \frac{1}{n} = y_n$ for all $n \in \bbN$ then both $\sum x_n$ and $\sum y_n$ diverge, but the product $\sum x_ny_n = \sum \frac{1}{n^2}$ which converges by the p-series test.
    \item [2.7.7] Now that we have proved the basic facts about geometric series, supply a proof for Corollary 2.4.7: The series $\sum_{n=1}^{\infty} \frac{1}{n^p}$ converges if and only if $p > 1$.
        \begin{proof}
            \begin{spacing}{1.2}
                We know from the Cauchy Condensation Test that if $(b_n)$ is a positive decreasing sequence then $\sum_{n=1}^{\infty}b_n$ converges if and only if $\sum_{n=1}^{\infty}2^n b_{2^n}$ converges.
                Suppose the sequence $(b_n)$ is the sequence of terms $\frac{1}{n^p}$ for all $n\in \bbN$.
                Then, $\sum_{n=1}^{\infty}b_n = \sum_{n=1}^{\infty} \frac{1}{n^p}$ converges if and only if $\sum_{n=1}^{\infty}2^n b_{2^n} = \sum_{n=1}^{\infty}2^n \frac{1}{(2^n)^p} = 
                \sum_{n=1}^{\infty}2^n (\frac{1}{2^n})^p = \sum_{n=1}^{\infty}(\frac{1}{2^n})^{p-1} = \sum_{n=1}^{\infty}(\frac{1}{2^{p-1}})^n$ converges. 
                $\sum_{n=1}^{\infty}(\frac{1}{2^{p-1}})^n$ is a geometric series which means that it converges if and only if $\abs{\frac{1}{2^{p-1}}} < 1$ which simplifies to $\abs{2^{p-1}} > 1$ which means $p > 1$.
            \end{spacing}
                Thus, the series $\sum_{n=1}^{\infty} \frac{1}{n^p}$ converges if and only if $p > 1$.
        \end{proof}
    \item [2.7.13] \textbf{(Dirichlet’s Test).} Dirichlet’s Test for convergence states that if the partial sums of $\sum_{n=1}^{\infty} x_n$ are bounded (but not necessarily convergent), and if $(y_n)$ is a sequence satisfying $y_1 \geq y_2 \geq y_3 \geq ... \geq 0$ with lim $y_n = 0$, then the series $\sum_{n=1}^{\infty} x_n y_n$ converges.
        \begin{itemize}
            \item [(a)] Let $M > 0$ be an upper bound for the partial sums of $\sum_{n=1}^{\infty}x_n$. Use Exercise 2.7.12 to show that $\abs{\sum_{j=m+1}^{n} x_j y_j} \leq 2M \abs{y_{m+1}}$.
                \begin{proof}
                    Let $s_n = x_1 + x_2 + ... + x_n$ and $M > 0$ be an upper bound for $s_n$. Then,
                    \begin{align*}
                        \abs{\sum_{j={m+1}}^n x_jy_j} &= \abs{s_ny_{n+1} - s_my_{m+1} + \sum_{j=m+1}^n s_j(y_j - y_{j+1})}\\
                        &\leq \abs{s_ny_{n+1} - s_my_{m+1}} + \abs{\sum_{j=m+1}^n s_j(y_j - y_{j+1})}\\
                        &\leq s_ny_{n+1} + s_my_{m+1} + \sum_{j=m+1}^n s_j(y_j - y_{j+1})\\
                        &\leq My_{n+1} + My_{m+1} + \sum_{j=m+1}^n M(y_j - y_{j+1})\\
                        &\leq My_{n+1} + My_{m+1} + M(y_{m+1} - y_{n+1})\\
                        &= 2My_{m+1}
                    \end{align*}
                    So, $\abs{\sum_{j=m+1}^{n} x_j y_j} \leq 2M \abs{y_{m+1}}$ as desired.
                \end{proof}
            \item [(b)] Prove Dirichlet’s Test just stated.
                \begin{proof}
                    Assume the partial sums of $\sum_{n=1}^{\infty} x_n$ are bounded and $(y_n)$ is a sequence satisfying $y_1 \geq y_2 \geq ... \geq 0$ with lim $y_n = 0$.
                    Let $M > 0$ be an upper bound for the partial sums of $\sum_{n=1}^{\infty}x_n$.
                    Then, from part a, $2M \abs{y_{m+1}} \geq \abs{\sum_{j=m+1}^{n} x_j y_j} = \abs{x_{m+1} y_{m+1} + x_{m+2} y_{m+2} + ... + x_{n} y_{n}}$.
                    Let $\epsilon > 0$.
                    Since, $\lim y_n = 0$, there exists an $N \in \bbN$ such that for all $m \geq N$, $\abs{y_m} < \frac{\epsilon}{2M}$.
                    Then, for all $n > m \geq N$, $\abs{x_{m+1} y_{m+1} + x_{m+2} y_{m+2} + ... + x_{n} y_{n}} \leq 2M \abs{y_{m+1}} < 2M \frac{\epsilon}{2M} = \epsilon$ which satisfies the Cauchy Criterion for Series.
                    Thus, the series $\sum_{n=1}^{\infty} x_n y_n$ converges.
                \end{proof}
            \item [(c)] Show how the Alternating Series Test (Theorem 2.7.7) can be derived as a special case of Dirichlet’s Test.
                
                Let $x_n = (-1)^{n+1}$ and $y_n = a_n$.
                Then, the partial sums of $\sum_{n=1}^{\infty} x_n$ are bounded and $(y_n)$ is a sequence satisfying $y_1 \geq y_2 \geq ... \geq 0$ with lim $y_n = 0$ which means Dirichlet's Test can be applied so $\sum_{n=1}^{\infty} x_n y_n = \sum_{n=1}^{\infty} (-1)^{n+1} a_n$ converges.
        \end{itemize}
\end{itemize}
\end{document}
\documentclass[12pt,letterpaper]{article}
\usepackage{preamble}
\usepackage{amsmath, amssymb}
\usepackage{mathtools}
% \DeclarePairedDelimiter{\abs}{\lvert}{\rvert}

\newcommand\course{311 Self Study}
\newcommand\hwnumber{2}
\newcommand\userID{Rohit Rao}

\begin{document}
\begin{itemize}[leftmargin=!,labelindent=5pt]
    \item [2.2.1] Verify, using the definition of convergence of a sequence, that the following sequences converge to the proposed limit.
        \begin{enumerate}
            \item lim $\frac{1}{6n^2 + 1}=0$
                \begin{proof}
                    Let $\epsilon > 0$ be arbitrary.
                    Choose $N \in \mathbb{N}$ with $N > \frac{1}{\sqrt{6\epsilon}}$.
                    To verify let $n \geq N$ which means $\abs{\frac{1}{6n^2 + 1}} < \abs{\frac{1}{6\frac{1}{6\epsilon} + 1}} = \abs{\frac{1}{\frac{1}{\epsilon} + 1}} = \frac{\epsilon}{\epsilon + 1} < \epsilon$ as desired.
                \end{proof}
            \item lim $\frac{3n+1}{2n + 5}=\frac{3}{2}$
                \begin{proof}
                    Let $\epsilon > 0$ be arbitrary.
                    Choose $N \in \mathbb{N}$ with $N > \frac{13-10\epsilon}{4\epsilon}$.
                    To verify let $n \geq N$ and so $\abs{\frac{3n+1}{2n + 5}} < \abs{\frac{3(\frac{13-10\epsilon}{4\epsilon})+1}{2(\frac{13-10\epsilon}{4\epsilon}) + 5}}= \frac{3}{2}-\epsilon$.
                    So, $\abs{\frac{3n+1}{2n + 5}-\frac{3}{2}}< \epsilon$ as desired.
                \end{proof}
            \item lim $\frac{2}{\sqrt{n+3}}=0$
                \begin{proof}
                    Let $\epsilon > 0$ be arbitrary. Choose $N \in \mathbb{N}$ with $N > \frac{4}{\epsilon^2}-3$. To verify let $n \geq N$ which means $\abs{\frac{2}{\sqrt{n+3}}}<\abs{\frac{2}{\sqrt{(\frac{4}{\epsilon^2}-3)+3}}}=\epsilon$ as desired.
                \end{proof}
        \end{enumerate}
    \item [2.2.7] Informally speaking, the sequence $\sqrt{n}$ “converges to infinity.”
        \begin{itemize}
            \item [(a)] Imitate the logical structure of Definition 2.2.3 to create a rigorous definition for the mathematical statement lim $x_n = \infty$. Use this definition to prove lim $\sqrt{n} = \infty$. 
            
                A sequence $(a_n)$ “converges to infinity” if, for every positive number $\epsilon$, there exists an $N \in \mathbb{N}$ such that whenever $n \geq N$ it follows that $a_n > \epsilon$.
                
                \begin{proof}
                    Let $\epsilon > 0$ be arbitrary. Choose $N \in \mathbb{N}$ with $N > \epsilon^2$. To verify let $n \geq N$ which means $\sqrt{n} > \sqrt{\epsilon^2} = \epsilon$ as desired.
                \end{proof}
            \item [(b)] What does your definition in (a) say about the particular sequence $(1,0,2,0,3,0,4,0,5,0,...)$?
            
            It does not converge to infinity.
        \end{itemize}
    \item [2.2.8] Here are two useful definitions:
        \begin{enumerate}
            \item A sequence $(a_n)$ is \textit{eventually} in a set $A \subseteq \mathbb{R}$ if there exists an $N \in \mathbb{N}$ such that $a_n \in A$ for all $n \geq N$.
            \item A sequence $(a_n)$ is \textit{frequently} in a set $A \subseteq \mathbb{R}$ if for every $N \in \mathbb{N}$, there exists an $n \geq N$ such that $a_n \in A$.
        \end{enumerate}
        \begin{itemize}
            \item [(a)] Is the sequence $(-1)^n$ eventually or frequently in the set $\{1\}$?
            
            Frequently
            \item [(b)] Which definition is stronger? Does frequently imply eventually or does eventually imply frequently?
            
            Eventually is stronger; eventually implies frequently.
            \item [(c)] Give an alternate rephrasing of Definition 2.2.3B using either frequently or eventually. Which is the term we want?
            
            A sequence $(a_n)$ converges to $a$ if, given any $\epsilon$-neighborhood $V_\epsilon(a)$ of $a$, $(a_n)$ is eventually in the set $V_\epsilon(a)$.
            \item [(d)] Suppose an infinite number of terms of a sequence $(x_n)$ are equal to 2. Is $(x_n)$ necessarily eventually in the interval $(1.9, 2.1)$? Is it frequently in $(1.9, 2.1)$?
            
            $x_n$ is not necessarily eventually in the interval $(1.9, 2.1)$, but it is frequently in $(1.9, 2.1)$. 
            For example, $(2,2,2,2,...)$ is eventually in the interval $(1.9, 2.1)$, but $(2,-2,2,-2,...)$ is only frequently in the interval $(1.9, 2.1)$.
        \end{itemize}
    \item [2.3.3] \textbf{(Squeeze Theorem)}. Show that if $x_n \leq y_n \leq z_n$ for all $n \in \mathbb{N}$, and if lim $x_n =$ lim $z_n = l$, then lim $y_n = l$ as well.
        \begin{proof}
            Assume that $\forall n \in \mathbb{N}, x_n \leq y_n \leq z_n$ and $\lim x_n = \lim z_n = l$.
            Suppose $n \in \mathbb{N}$.
            Then, by the Order Limit Theorem, since $x_n \leq y_n$, $\lim y_n \geq \lim x_n = l$.
            Similarly, since $y_n \leq z_n$, $\lim y_n \leq \lim z_n = l$.
            So, $l \leq \lim y_n \leq l$.
            Thus, $\lim y_n = l$.
        \end{proof}
    \item [2.3.5] Let $(x_n)$ and $(y_n)$ be given, and define $(z_n)$ to be the “shuffled” sequence \\$(x_1, y_1, x_2, y_2, x_3, y_3, . . . , x_n, y_n, . . . )$. Prove that $(z_n)$ is convergent if and only if $(x_n)$ and $(y_n)$ are both convergent with lim $x_n =$ lim $y_n$.
        \begin{proof}
            \ \\
            First we prove that if $(z_n)$ is convergent then $(x_n)$ and $(y_n)$ are both convergent with $\lim x_n = \lim y_n$.
            Assume $(z_n)$ is convergent to $z$.
            Let $\epsilon > 0$ be arbitrary.
            Because $(z_n)$ is convergent to $z$ we know there exists an $N \in \bbN$ such that for all $n \geq N$, $\abs{z_n - z} < \epsilon$.
            So, $\epsilon > \abs{x_n-z}$ for all $n \geq N$ because $x_n = z_{2n-1}$ and $2n-1 \geq N$.
            Similarly, $\epsilon > \abs{y_n-z}$ for all $n \geq N$ because $y_n = z_{2n}$ and $2n \geq N$.
            Therefore, $\lim x_n = z = \lim y_n$.

            Next we prove that if $(x_n)$ and $(y_n)$ are both convergent with $\lim x_n = \lim y_n$ then $(z_n)$ is convergent.
            Assume $(x_n)$ and $(y_n)$ are both convergent with $\lim x_n = \lim y_n = l$.
            Let $\epsilon > 0$ be arbitrary.
            Then, we know that there exists an $N_1 \in \bbN$ such that for all $n \geq N_1$, $\abs{x_n - l} < \epsilon$.
            Similarly, we know that there exists an $N_2 \in \bbN$ such that for all $n \geq N_2$, $\abs{y_n - l} < \epsilon$.
            Let $N = $ max $\{2N_1, 2N_2\}$.
            Then, for all $n \geq N$, we have that $\abs{z_n - l} < \epsilon$ since $(z_n)$ consists of alternating elements from $(x_n)$ and $(y_n)$ which means after $N$ both $\abs{x_n - l} < \epsilon$ and $\abs{y_n - l} < \epsilon$.
            Thus, $(z_n)$ is convergent.
        \end{proof}
    \newpage
    \item [2.3.10] If $(a_n) \to 0$ and $\abs{b_n - b} \leq a_n$, then show that $(b_n) \to b$.
        \begin{proof}
            Assume $(a_n) \to 0$ and $\abs{b_n - b} \leq a_n$. 
            Let $\epsilon > 0$ be arbitrary.
            Since $(a_n) \to 0$, we know that there exists $N \in \bbN$ such that for all $n \geq N, \abs{a_n} < \epsilon$.
            Then, for all $n \geq N$, $\abs{b_n - b} \leq a_n < \epsilon$ which means $\abs{b_n - b} < \epsilon$, and so $(b_n) \to b$.
        \end{proof}
\end{itemize}
\end{document}
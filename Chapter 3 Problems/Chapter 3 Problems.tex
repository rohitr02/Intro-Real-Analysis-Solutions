\documentclass[12pt,letterpaper]{article}
\usepackage{preamble}

\newcommand\course{311 Self Study}
\newcommand\hwnumber{3}
\newcommand\userID{Rohit Rao}

\begin{document}
    \begin{itemize}[leftmargin=!,labelindent=5pt]
        \item [3.2.2] Let $B = \{\frac{(-1)^nn}{n+1} : n = 1, 2, 3, ...\}$.
            \begin{itemize}
                \item [(a)] Find the limit points of $B$.
                
                    $\{-1, 1\}$
                \item [(b)] Is $B$ a closed set?
                
                    No, it contains neither of its limit points.
                \item [(c)] Is $B$ an open set?
                
                    No, its not possible to find an $\epsilon$-neighborhood for every point in $B$ such that the $\epsilon$-neighborhood is contained in $B$.
                \item [(d)] Does $B$ contain any isolated points?
                
                    Every element of $B$ is an isolated point.
                \item [(e)] Find $\overline{B}$.
                
                    $B \cup \{-1, 1\}$
            \end{itemize}
        \item [3.2.6] Prove Theorem 3.2.8: A set $F \subseteq \bbR$ is closed if and only if every Cauchy sequence contained in $F$ has a limit that is also an element of $F$.
            \begin{proof}
                \ \\
                First we prove that if a set $F \subseteq \bbR$ is closed then every Cauchy sequence contained in $F$ has a limit that is also an element of $F$.
                Assume $F \subseteq \bbR$ is closed, that is $F$ contains its limit points.
                So, we need to show that every Cauchy sequence $(a_n)$ contained in $F$ has a limit in $F$.
                Assume $(a_n)$ is an arbitrary Cauchy sequence contained in $F$.
                Since $(a_n)$ is Cauchy, it's limit exists.
                So, let $a = \lim (a_n)$.
                Now, we need to show that $a$ is either a limit point in $F$ or an isolated point in $F$.
                If $a_n \neq a$ for all $n \in \bbN$, then $a$ is a limit point and since $F$ is closed, $a \in F$.
                Otherwise, $a_n = a$ for some $n \in \bbN$, and since $(a_n) \subseteq F$, $a \in F$.
                So, every Cauchy sequence contained in $F$ has a limit that is also an element of $F$.
                \\
                Next, we prove that if every Cauchy sequence contained in a set $F$ has a limit that is also an element of $F$, then $F \subseteq \bbR$ is closed.
                Assume every Cauchy sequence contained in a set $F \subseteq \bbR$ has a limit that is also an element of $F$.
                Need to show that $F$ is closed, that is $F$ contains all its limit points.
                Let $a$ be an arbitrary limit point of $F$.
                Then, $a = \lim a_n$ for some sequence $(a_n)$ contained in $F$.
                Since $(a_n)$ converges, it must be a Cauchy sequence.
                So, $a \in F$. Thus, $F$ is closed.
            \end{proof}
        \newpage
        \item [3.2.10] (De Morgan's Laws): A proof for De Morgan’s Laws in the case of two sets is outlined in Exercise 1.2.3. The general argument is similar.
            \begin{itemize}
                \item [(a)] Given a collection of sets $\{E_\lambda : \lambda \in \Lambda\}$, show that $(\bigcup_{\lambda \in \Lambda} E_\lambda)^c = \bigcap_{\lambda \in \Lambda} E_{\lambda}^c$ and $(\bigcap_{\lambda \in \Lambda} E_\lambda)^c = \bigcup_{\lambda \in \Lambda} E_{\lambda}^c$.
                    \begin{spacing}{1.175}
                        \begin{proof}
                            First, we need to show that $(\bigcup_{\lambda \in \Lambda} E_\lambda)^c \subseteq \bigcap_{\lambda \in \Lambda} E_{\lambda}^c$, that is $\forall x \in (\bigcup_{\lambda \in \Lambda} E_\lambda)^c$, $x \in \bigcap_{\lambda \in \Lambda} E_{\lambda}^c$.
                            Suppose $x \in (\bigcup_{\lambda \in \Lambda} E_\lambda)^c$.
                            Then, by definition of set complement, $\forall \lambda \in \Lambda, x \notin E_\lambda$.
                            So, $\forall \lambda \in \Lambda, x \in E_\lambda^c$.
                            Then, by definition of set intersection, $x \in \bigcap_{\lambda \in \Lambda} E_{\lambda}^c$.
                            So, $(\bigcup_{\lambda \in \Lambda} E_\lambda)^c \subseteq \bigcap_{\lambda \in \Lambda} E_{\lambda}^c$.
                            Next, we need to show that $\bigcap_{\lambda \in \Lambda} E_{\lambda}^c \subseteq (\bigcup_{\lambda \in \Lambda} E_\lambda)^c$, that is $\forall y \in \bigcap_{\lambda \in \Lambda} E_{\lambda}^c$, $y \in (\bigcup_{\lambda \in \Lambda} E_\lambda)^c$.
                            Suppose $y \in \bigcap_{\lambda \in \Lambda} E_{\lambda}^c$.
                            Then, by definition of set intersection, $\forall \lambda \in \Lambda, y \in E_\lambda^c$.
                            So, $\forall \lambda \in \Lambda, y \notin E_\lambda$.
                            Then, by definition of set union, $y \notin \bigcup_{\lambda \in \Lambda} E_\lambda$.
                            So, $y \in (\bigcup_{\lambda \in \Lambda} E_\lambda)^c$.
                            Thus, $(\bigcup_{\lambda \in \Lambda} E_\lambda)^c \subseteq \bigcap_{\lambda \in \Lambda} E_{\lambda}^c$ and $\bigcap_{\lambda \in \Lambda} E_{\lambda}^c \subseteq (\bigcup_{\lambda \in \Lambda} E_\lambda)^c$ means that $(\bigcup_{\lambda \in \Lambda} E_\lambda)^c = \bigcap_{\lambda \in \Lambda} E_{\lambda}^c$.
                        \end{proof}

                        \begin{proof}
                            First, we need to show that $(\bigcap_{\lambda \in \Lambda} E_\lambda)^c \subseteq \bigcup_{\lambda \in \Lambda} E_{\lambda}^c$, that is $\forall x \in (\bigcap_{\lambda \in \Lambda} E_\lambda)^c$, $x \in \bigcup_{\lambda \in \Lambda} E_{\lambda}^c$.
                            Suppose $x \in (\bigcap_{\lambda \in \Lambda} E_\lambda)^c$.
                            So, $x \notin \bigcap_{\lambda \in \Lambda} E_\lambda$ which means that there exists at least one $\lambda' \in \Lambda$ such that $x \notin E_{\lambda'}$.
                            Choose $\lambda' \in \Lambda$ such that $x \notin E_{\lambda'}$.
                            Then, $x \in E_{\lambda'}^c$.
                            So, $x \in \bigcup_{\lambda \in \Lambda} E_{\lambda}^c$ which means $(\bigcap_{\lambda \in \Lambda} E_\lambda)^c \subseteq \bigcup_{\lambda \in \Lambda} E_{\lambda}^c$.
                            Next we need to prove that $\bigcup_{\lambda \in \Lambda} E_{\lambda}^c \subseteq (\bigcap_{\lambda \in \Lambda} E_\lambda)^c$, that is $\forall y \in \bigcup_{\lambda \in \Lambda} E_{\lambda}^c$, $y \in (\bigcap_{\lambda \in \Lambda} E_\lambda)^c$.
                            Suppose $y \in \bigcup_{\lambda \in \Lambda} E_{\lambda}^c$.
                            Then, there exists at least one $\lambda'' \in \Lambda$ such that $y \notin E_{\lambda''}$.
                            Choose $\lambda'' \in \Lambda$ such that $y \notin E_{\lambda''}$.
                            Then, $y \notin \bigcap_{\lambda \in \Lambda} E_{\lambda}$.
                            So, $y \in (\bigcap_{\lambda \in \Lambda} E_{\lambda})^c$ which means $\bigcup_{\lambda \in \Lambda} E_{\lambda}^c \subseteq (\bigcap_{\lambda \in \Lambda} E_\lambda)^c$.
                            Thus, $(\bigcap_{\lambda \in \Lambda} E_\lambda)^c \subseteq \bigcup_{\lambda \in \Lambda} E_{\lambda}^c$ and $\bigcup_{\lambda \in \Lambda} E_{\lambda}^c \subseteq (\bigcap_{\lambda \in \Lambda} E_\lambda)^c$ means that $(\bigcap_{\lambda \in \Lambda} E_\lambda)^c = \bigcup_{\lambda \in \Lambda} E_{\lambda}^c$.
                        \end{proof}
                    \end{spacing}
                \item [(b)] Now, provide the details for the proof of Theorem 3.2.14
                
                (i) The union of a finite collection of closed sets is closed.
                    \begin{proof}
                        Suppose $\{E_\lambda : \lambda \in \Lambda\}$ is a collection of closed sets.
                        Then, $\{E_\lambda : \lambda \in \Lambda\}^c$ is a collection of open sets and we know that the intersection of a finite amount of open sets is open (Theorem 3.2.3).
                        So, taking the complement again $(\{E_\lambda : \lambda \in \Lambda\}^c)^c = \{E_\lambda : \lambda \in \Lambda\}$ gives us a closed set (since the complement of an open set is a closed set) as desired.
                    \end{proof}
                (ii) The intersection of an arbitrary collection of closed sets is closed.
                    \begin{proof}
                        Suppose $\{E_\lambda : \lambda \in \Lambda\}$ is an arbitrary collection of closed sets.
                        Then, $E_{\lambda}^c$ is open and $\forall \lambda \in \Lambda$, the union of $E_{\lambda}^c$ is open (Theorem 3.2.3).
                        By De Morgan's Law, we know $\bigcup_{\lambda \in \Lambda} E_{\lambda}^c = (\bigcap_{\lambda \in \Lambda} E_\lambda)^c$ so $(\bigcap_{\lambda \in \Lambda} E_\lambda)^c$ is open.
                        Then, $\bigcap_{\lambda \in \Lambda} E_\lambda$ is closed.
                        Thus, the intersection of an arbitrary collection of closed sets is closed.
                    \end{proof}
            \end{itemize}
        \item [3.3.4] Show that if $K$ is compact and $F$ is closed, then $K \cap F$ is compact.
        
            \begin{proof}
                Suppose $F$ is closed and $K$ is compact, that is $K$ is bounded and closed.
                Need to show that $K \cap F$ is compact, that is $K \cap F$ is bounded and closed.
                Since $K$ is bounded and $K \cap F \subseteq K$, $K \cap F$ is bounded.
                Also, since $K$ is closed and $F$ is closed, $K \cap F$ is closed (since the intersection of an arbitrary collection of closed sets is closed).
                So, $K \cap F$ is bounded and closed which means $K \cap F$ is compact.
            \end{proof}
        \item [3.3.8] Follow these steps to prove the final implication in Theorem 3.3.8.
            
            Assume $K$ satisfies (i) and (ii), and let $\{O_\lambda : \lambda \in \Lambda\}$ be an open cover for $K$. For contradiction, let’s assume that no finite subcover exists. Let $I_0$ be a closed interval containing $K$, and bisect $I_0$ into two closed intervals $A_1$ and $B_1$.
            \begin{itemize}
                \item [(a)] Why must either $A_1 \cap K$ or $B_1 \cap K$ (or both) have no finite subcover consisting of sets from $\{O_\lambda : \lambda \in \Lambda\}$.
                
                    At least one of $A_1 \cap K$ or $B_1 \cap K$ must have no finite subcover since if they both did have a finite subcover then the union of them would be a finite subcover for $K$ which would contradict the assumption that no finite subcover exists for $K$.
                \item [(b)] Show that there exists a nested sequence of closed intervals $I_0 \supseteq I_1 \supseteq I_2 \supseteq ... $ with the property that, for each $n$, $I_n \cap K$ cannot be finitely covered and $\lim \abs{I_n} = 0$.
                
                    Choose whichever of $A_1 \cap K$ or $B_1 \cap K$ does not have a finite subcover (choose any one if they both do not), then call that choice $I_1$.
                    Then, bisect $I_1$ to give $A_2$ and $B_2$.
                    Once again, either $A_2 \cap K$ or $B_2 \cap K$ (or both) have no finite subcover.
                    Choose whichever of $A_2 \cap K$ or $B_2 \cap K$ does not have a finite subcover (choose either one if they both do not), then we can call that choice $I_2$.
                    Repeating this over results in the sequence $I_0 \supseteq I_1 \supseteq I_2 \supseteq ... $ where $I_n \cap K$ cannot be finitely covered and as this sequence goes further, it tends towards $\lim \abs{I_n} = 0$.
                \item [(c)] Show that there exists an $x \in K$ such that $x \in I_n$ for all $n$.
                
                    Since $K$ is compact, it is closed and bounded.
                    So, $K \cap I_n \subseteq K$ is also closed and bounded for all $n \in \bbN$ which means $K \cap I_n$ is compact.
                    Thus, by Theorem 3.3.5, the intersection of a nested sequence of nonempty compact sets is nonempty, that is $\exists x \in K$ such that $x \in K \cap I_n \subseteq I_n$ for all $n \in \bbN$.
                    So, $x \in I_n$ for all $n \in \bbN$.

                \item [(d)] Because $x \in K$, there must exist an open set $O_{\lambda_0}$ from the original collection that contains $x$ as an element. Argue that there must be an $n_0$ large enough to guarantee that $I_{n_0} \subseteq O_{\lambda_0}$. Explain why this furnishes us with the desired contradiction.
                
                    Since $O_{\lambda_0}$ is an open set, there exists $\epsilon > 0$ such that the $\epsilon$-neighborhood $V_\epsilon(x) \subseteq O_{\lambda_0}$.
                    So, choose $n_0 \in \bbN$ such that $\abs{I_{n_0}} < \epsilon$.
                    Then, $I_{n_0} \subseteq O_{\lambda_0}$ which means $I_{n_0}$ has a finite subcover.
                    However, this is a contradiction to the initial claim that $K$ has no finite subcover
                    because $K \cap I_{n_0}$ has a finite subcover, namely $O_{\lambda_0}$.
            \end{itemize}
        \item [3.3.10] Let’s call a set clompact if it has the property that every closed cover (i.e., a cover consisting of closed sets) admits a finite subcover. Describe all of the clompact subsets of $\bbR$.
            
            All finite sets in $\bbR$ are clompact.
        \item [3.4.4] 
        \item [3.4.5] 
        \item [3.4.7] 
        \item [3.5.1] 
        \item [3.5.2]
        \item [3.5.3]
    \end{itemize}
\end{document}
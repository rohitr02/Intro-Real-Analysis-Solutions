\documentclass[12pt,letterpaper]{article}
\usepackage{preamble}

\newcommand\course{311 Self Study}
\newcommand\hwnumber{4}
\newcommand\userID{Rohit Rao}

\begin{document}
    \begin{itemize}[leftmargin=!,labelindent=5pt]
        \item [4.2.2] Assume a particular $\delta > 0$ has been constructed as a suitable response to a particular $\epsilon$ challenge. Then, any larger/smaller (pick one) $\delta$ will also suffice.
        
            Smaller.
        \item [4.2.4] Review the definition of Thomae’s function $t(x)$ from Section 4.1.
        
            \[ t(x) =
            \begin{cases}
                1 & \text{if } x = 0\\
                \frac{1}{n} & \text{if } x = \frac{m}{n} \in \bbQ - \{0\} \text{ is in lowest terms with } n > 0\\
                0 & \text{if } x \notin \bbQ
            \end{cases}
            \]
            \begin{itemize}
                \item [(a)] Construct three different sequences $(x_n)$, $(y_n)$, and $(z_n)$, each of which converges to 1 without using the number 1 as a term in the sequence.
                
                    $(x_n) = 1 - \frac{1}{n}$ for all $n \in N$ such that $n > 1$.

                    $(y_n) = 1 + \frac{1}{n}$ for all $n \in N$ such that $n > 1$.

                    $(z_n) = 1 - \frac{1}{n^2}$ for all $n \in N$ such that $n > 1$.
                \item [(b)] Now, compute $\lim t(x_n)$, $\lim t(y_n)$, and $\lim t(z_n)$.
                
                    $\lim t(x_n) = \lim t(y_n) = \lim t(z_n) = 0$
                \item [(c)] Make an educated conjecture for $\lim_{x \to 1} t(x)$, and use Definition 4.2.1B to verify the claim. (Given $\epsilon > 0$, consider the set of points $\{x\in \bbR : t(x) \geq \epsilon\}$. Argue that all the points in this set are isolated.)
                
                    My conjecture is that $\lim_{x \to 1} t(x) = 0$.
                    \begin{proof}
                        Suppose $\epsilon > 0$.
                        Consider the set $T = \{x\in \bbR : t(x) \geq \epsilon\}$.
                        From the way $T$ is defined, we know that $x \in T$ means that either $x=0$ or $x=\frac{m}{n}$ in simplest form where $m,n \in \bbQ - \{0\}$ with $n>0$.
                        Since $x$ approaches 1 in our limit, $x \in T$ is of the latter form. 
                        So, $x = \frac{m}{n}$ which means $t(x) = \frac{1}{n} \geq \epsilon$ which simplifies to $n \leq \frac{1}{\epsilon}$.
                        Given some arbitrary finite interval $(0,a]$ where $a \in \bbR$, the set $T \cap (0,a]$ is also finite since there are only a finite amount of rational numbers $\frac{m}{n}$ with $n \leq \frac{1}{\epsilon}$.
                        Since this set is finite choose $\delta = \min{\{y : y \in T \cap (0,a]\}}$.
                        We can see that this choice of $\delta$ is correct because choosing $x \in V_\delta(1)$ means $x \notin T$ which means $t(x) \in V_\epsilon(0)$.
                    \end{proof}
            \end{itemize}
        \item [4.2.5] 
            \begin{itemize}
                \item [(a)] Supply the details for how Corollary 4.2.4 part (ii) follows from the sequential criterion for functional limits in Theorem 4.2.3 and the Algebraic Limit Theorem for sequences proved in Chapter 2.
                    \begin{proof}
                        Suppose $f$ and $g$ are functions defined on the domain $A \subseteq \bbR$.
                        Assume $\lim_{x \to c} f(x) = L$ and $\lim_{x \to c} g(x) = M$ for some limit point $c$ of $A$.
                        Then, we know from the Sequential Criterion for Functional Limits (Theorem 4.2.3) that for all sequences $(x_n) \subseteq A$ satisfying $x_n \neq c$ and $(x_n) \to c$, it follows that $f(x_n) \to L$ and $g(x_n) \to M$.
                        Then, using the algebraic limit theroem for sequences, we know that $f(x_n) + g(x_n) \to L+M$.
                        Thus, $\lim_{x \to c} [f(x) + g(x)] = L + M$ as desired.
                    \end{proof}
                \item [(b)] Now, write another proof of Corollary 4.2.4 part (ii) directly from Definition 4.2.1 without using the sequential criterion in Theorem 4.2.3.
                    \begin{proof}
                        Suppose $f$ and $g$ are functions defined on the domain $A \subseteq \bbR$.
                        Assume $\lim_{x \to c} f(x) = L$ and $\lim_{x \to c} g(x) = M$ for some limit point $c$ of $A$.
                        Let $\epsilon > 0$.
                        Need to show that $\lim_{x \to c} [f(x) + g(x)] = L + M$, that is there exists $\delta > 0$ such that whenever $x \in A$ and $0 < \abs{x-c} < \delta$, it follows that $\abs{(f(x) + g(x)) - L - M} < \epsilon$.
                        Since $\lim_{x \to c} f(x) = L$, we know that there exists a $\delta_1>0$ such that whenever $x \in A$ and $0 < \abs{x-c} < \delta_1$, it follows that $\abs{f(x) - L} < \frac{\epsilon}{2}$.
                        Similarly, since $\lim_{x \to c} g(x) = M$, we know that there exists a $\delta_2>0$ such that whenever $x \in A$ and $0 < \abs{x-c} < \delta_2$, it follows that $\abs{g(x) - M} < \frac{\epsilon}{2}$.
                        So, choose $\delta$ such that $\delta = \min\{\delta_1, \delta_2\}$.
                        Then, whenever $0 < \abs{x-c} < \delta$:
                        \begin{align*}
                            \abs{(f(x) + g(x)) - (L + M)} &= \abs{(f(x) - L) + (g(x) - M)} \\
                            &\leq \abs{(f(x) - L)} + \abs{(g(x) - M)}\\
                            &< \frac{\epsilon}{2} + \frac{\epsilon}{2} = \epsilon.
                        \end{align*}
                        So, when $x \in A$ and $0 < \abs{x-c} < \delta$, $\abs{(f(x) + g(x)) - (L + M)} < \epsilon$ as desired.
                    \end{proof}

                \item [(c)] Repeat (a) and (b) for Corollary 4.2.4 part (iii).
                        \begin{proof}
                            Suppose $f$ and $g$ are functions defined on the domain $A \subseteq \bbR$.
                            Assume $\lim_{x \to c} f(x) = L$ and $\lim_{x \to c} g(x) = M$ for some limit point $c$ of $A$.
                            Then, we know from the Sequential Criterion for Functional Limits (Theorem 4.2.3) that for all sequences $(x_n) \subseteq A$ satisfying $x_n \neq c$ and $(x_n) \to c$, it follows that $f(x_n) \to L$ and $g(x_n) \to M$.
                            Then, using the algebraic limit theroem for sequences, we know that $f(x_n) * g(x_n) \to LM$.
                            Thus, $\lim_{x \to c} [f(x)g(x)] = LM$ as desired.
                        \end{proof}

                        \begin{proof}
                            Suppose $f$ and $g$ are functions defined on the domain $A \subseteq \bbR$.
                            Assume $\lim_{x \to c} f(x) = L$ and $\lim_{x \to c} g(x) = M$ for some limit point $c$ of $A$.
                            Let $\epsilon > 0$.
                            Need to show that $\lim_{x \to c} [f(x)g(x)] = LM$, that is there exists $\delta > 0$ such that whenever $x \in A$ and $0 < \abs{x-c} < \delta$, it follows that $\abs{(f(x)g(x)) - LM} < \epsilon$.
                            Simplifying further, we need that there exists $\delta > 0$ such that whenever $x \in A$ and $0 < \abs{x-c} < \delta$,
                                \begin{align*}
                                    \abs{(f(x)g(x)) - LM} &= \abs{f(x)g(x) - Mf(x) + Mf(x) - LM} \\
                                    &\leq \abs{f(x)}\abs{(g(x)-M)} + \abs{M}\abs{(f(x) - L)} < \epsilon.
                                \end{align*}
                            Since $\lim_{x \to c} f(x) = L$, we know that there exists a $\delta_1>0$ such that whenever $x \in A$ and $0 < \abs{x-c} < \delta_1$, it follows that $\abs{f(x) - L} < \frac{\epsilon}{2M}$ where $M \neq 0$.
                            Since $\lim_{x \to c} g(x) = M$, we know that there exists a $\delta_2>0$ such that whenever $x \in A$ and $0 < \abs{x-c} < \delta_2$, it follows that $\abs{g(x) - M} < \frac{\epsilon}{2(\abs{L}+1)}$.
                            Simplifying further, we need that there exists $\delta > 0$ such that whenever $x \in A$ and $0 < \abs{x-c} < \delta$,
                                \begin{align*}
                                    \abs{(f(x)g(x)) - LM} &\leq \abs{f(x)}\abs{(g(x)-M)} + \abs{M}\abs{(f(x) - L)} \\
                                    &< \abs{f(x)}\frac{\epsilon}{2(\abs{L}+1)} + M\frac{\epsilon}{2M}
                                \end{align*}
                            So, we need to show $\abs{f(x)} \leq \abs{L} + 1$.
                            Well, $\abs{f(x)} = \abs{f(x) - L + L} \leq \abs{f(x) - L} + \abs{L}$.
                            Since $\abs{f(x) - L}$ can get arbitrarily small, we know that there exists a $\delta_3 >0$ such that whenever $x \in A$ and $0 < \abs{x-c} < \delta_3$, it follows that $\abs{f(x) - L} < 1$.
                            So, $\abs{f(x) - L} + \abs{L} \leq 1 + \abs{L}$ which means $\abs{f(x)} \leq \abs{L} + 1$.
                            Now choose $\delta = \min{\{\delta_1, \delta_2, \delta_3\}}$.
                            Then, we have that
                                \begin{align*}
                                    \abs{(f(x)g(x)) - LM} &< \abs{f(x)}\frac{\epsilon}{2(\abs{L}+1)} + M\frac{\epsilon}{2M}\\
                                    &< (\abs{L}+1)\frac{\epsilon}{2(\abs{L}+1)} + M\frac{\epsilon}{2M}\\
                                    &= \frac{\epsilon}{2} + \frac{\epsilon}{2} = \epsilon
                                \end{align*}
                            whenever $x \in A$ and $0 < \abs{x - c} < \delta$ and $M \neq 0$.
                            So, $\lim_{x \to c} [f(x)g(x)] = LM$.
                        \end{proof}
            \end{itemize}
        \item [4.2.6] Let $g : A \to \bbR$ and assume that $f$ is a bounded function on $A \subseteq \bbR$ (i.e., there exists $M > 0$ satisfying $\abs{f(x)} \leq M$ for all $x \in A$). Show that if $\lim_{x \to c} g(x) = 0$, then $\lim_{x \to c} g(x)f(x) = 0$ as well.
            \begin{proof}
                Suppose $\epsilon > 0$.
                Need to show $\lim_{x \to c} g(x)f(x) = 0$, that is $\exists \delta > 0$ such that $x \in A$ and $0 < \abs{x-c} < \delta$ implies $\abs{f(x)g(x)} < \epsilon$.
                $\lim_{x \to c} g(x) = 0$ means that there exists $\delta_1>0$ such that $x \in A$ and $0 < \abs{x-c} < \delta_1$ implies $\abs{g(x)} < \frac{\epsilon}{M}$.
                Choose $\delta = \delta_1$.
                Then, since $\abs{f(x)} \leq M$ for all $x \in A$, we have that whenever $x \in A$ and $0 < \abs{x-c} < \delta$, $\abs{f(x)g(x)} < M *\frac{\epsilon}{M} = \epsilon$.
                Thus, $\lim_{x \to c} g(x)f(x) = 0$ as desired.
            \end{proof}
        \item [4.2.7] 
            \begin{itemize}
                \item [(a)] The statement $\lim_{x \to 0} \frac{1}{x^2} = \infty$ certainly makes intuitive sense. Construct a rigorous definition in the “challenge–response” style of Definition 4.2.1 for a limit statement of the form $\lim_{x \to c} f(x) = \infty$ and use it to prove the previous statement.

                    Definition: Let $f : A \to \bbR$. We say that $\lim_{x \to c} f(x) = \infty$ provided that, for all $N > 0$, there exists a $\delta > 0$ such that whenever $x \in A$ and $0 < \abs{x-c} < \delta$, it follows that $f(x) > N$. 
                    
                    \begin{proof}
                        Suppose $N > 0$.
                        Need that there exists a $\delta > 0$ such that whenever $x \in A$ and $0 < \abs{x-c} = \abs{x} < \delta$, it follows that $f(x) > N$.
                        Choose $\delta = \frac{1}{\sqrt(N)}$.
                        Then, whenever $x \in A$ and $0 < \abs{x} < \delta = \frac{1}{\sqrt(N)}$, it follows that $x^2 <  \frac{1}{N}$.
                        Thus, $\frac{1}{x^2} = f(x) > N$ as desired.
                    \end{proof}
                \item [(b)] Now, construct a definition for the statement $\lim_{x \to \infty} f(x) = L$. Show $\lim_{x \to \infty} \frac{1}{x} = 0$.
                    
                    Definition: Let $f : A \to \bbR$. We say that $\lim_{x \to \infty} f(x) = L$ provided that, for all $\epsilon > 0$, there exists a $D > 0$ such that whenever $x \in A$ and $x > D$, it follows that $\abs{f(x) - L} < \epsilon$. 

                    \begin{proof}
                        Suppose $\epsilon > 0$.
                        Need that there exists a $D > 0$ such that whenever $x \in A$ and $x > D$, it follows that $\abs{\frac{1}{x} - 0} < \epsilon$. 
                        Choose $D = \frac{1}{\epsilon}$.
                        Then, whenever $x \in A$ and $x > D = \frac{1}{\epsilon}$, it follows that $\abs{\frac{1}{x}} < \abs{\frac{1}{\frac{1}{\epsilon}}} = \abs{\epsilon} = \epsilon$.
                        Thus, $\abs{\frac{1}{x}} < \epsilon$ as desired.
                    \end{proof}
                \item [(c)] What would a rigorous definition for $\lim_{x \to \infty} f(x) = \infty$ look like? Give an example of such a limit.
                
                    Definition: Let $f : A \to \bbR$. We say that $\lim_{x \to \infty} f(x) = \infty$ provided that, for all $N > 0$, there exists a $D > 0$ such that whenever $x \in A$ and $x > D$, it follows that $f(x) > N$.

                    An example of such a limit would be $f(x) = x$ because $\lim_{x\to\infty} x = \infty$.

                    \begin{proof}
                        Suppose $N > 0$.
                        Need that there exists a $D > 0$ such that whenever $x \in A$ and $x > D$, it follows that $f(x) = x > N$.
                        Choose $D = N$.
                        Then, whenever $x \in A$ and $x > D$, it follows that $x > N$ as desired.
                    \end{proof}
            \end{itemize}
        \item [4.2.8] Assume $f(x) \geq g(x)$ for all $x$ in some set $A$ on which $f$ and $g$ are defined. Show that for any limit point $c$ of $A$ we must have $\lim_{x \to c} f(x) \geq \lim_{x \to c} g(x)$.
            \begin{proof}
                Assume $f(x) \geq g(x)$ for all $x$ in some set $A$ on which $f$ and $g$ are defined.
                Need to show that for any limit point $c$ of $A$, $\lim_{x \to c} f(x) \geq \lim_{x \to c} g(x)$.
                Suppose $c$ is any limit point of $A$.
                Let $ L = \lim_{x \to c} f(x)$ and $M = \lim_{x \to c} g(x)$.
                Then, we know from the Sequential Criterion for Functional Limits (Theorem 4.2.3) that for all sequences $(x_n) \subseteq A$ satisfying $x_n \neq c$ and $(x_n) \to c$, it follows that $f(x_n) \to L$ and $g(x_n) \to M$.
                We know that $f(x_n) \geq g(x_n)$ and so by the Order Limit Theorem we have that $\lim f(x_n) = L \geq \lim g(x_n) = M$.
                So, $\lim_{x \to c} f(x) \geq \lim_{x \to c} g(x)$ as desired.
            \end{proof}

        \item [4.3.2] 
            \begin{itemize}
                \item [(a)] Supply a proof for Theorem 4.3.9 using the $\epsilon / \delta$ characterization of continuity.
                    
                    Theorem 4.3.9: Given $f:A\to\bbR$ and $g:B\to\bbR$, assume that the range $f(A) = \{f(x): x \in A\}$ is contained in the domain $B$ so that the composition $g \circ f(x) = g(f(x))$ is well-defined on $A$.

                    If $f$ is continuous at $c \in A$, and if $g$ is continuous at $f(c) \in B$, then $g \circ f$ is continuous at $c$.
                    \begin{proof}
                        Suppose $\epsilon > 0$ is arbitrary.
                        Since $g$ is continuous at $f(c) \in B$, there exists a $\delta_1>0$ such that $\abs{g(f(x))-g(f(c))} < \epsilon$ whenever $\abs{f(x) - f(c)} < \delta_1$ (and $x \in A$ and $f(x) \in B$).
                        Since $f$ is continuous at $c \in A$, there exists a $\delta > 0$ such that $\abs{f(x) - f(c)} < \delta_1$ whenever $\abs{x-c} < \delta$ (and $x \in A$).
                        Combining both of the above parts, we see that for all $\epsilon > 0$, there exists a $\delta > 0$ such that $\abs{x-c} < \delta$ implies $\abs{f(x) - f(c)} < \delta_1$ which implies $\abs{g(f(x))-g(f(c))} < \epsilon$.
                        Thus, $g \circ f$ is continuous at $c$.
                    \end{proof}
                \item [(b)] Give another proof of this theorem using the sequential characterization of continuity (from Theorem 4.3.2 (iv)).
                    \begin{proof}
                        Assume that $(x_n) \to c$ (with $x_n \in A$).
                        Since $f$ is continuous at $c$, $f(x_n) \to f(c)$ by Theorem 4.3.2 (iv) (Characterizations of Continuity).
                        Since $g$ is continuous at $f(c)$, $g(f(x_n)) \to g(f(c))$ by Theorem 4.3.2 (iv).
                        Thus, $g \circ f$ is continuous at $c$.
                    \end{proof}
            \end{itemize}
        \item [4.3.4] 
            \begin{itemize}
                \item [(a)] Show using Definition 4.3.1 that any function $f$ with domain $\bbZ$ will necessarily be continuous at every point in its domain.
                    \begin{proof}
                        Suppose $f$ is a function with domain $\bbZ$.
                        Need to show that for all $\epsilon > 0$, there exists $\delta > 0$ such that whenever $\abs{x-c} < \delta$ (and $x,c \in \bbZ$) it follows that $\abs{f(x) - f(c)} < \epsilon$.
                        Suppose $\epsilon > 0$ and $c \in \bbZ$ is arbitrary.
                        Choose $\delta = 1$.
                        Then, we see that whenever $\abs{x-c} < \delta$ (and $x,c \in \bbZ$), it follows that $\abs{f(x) - f(c)} < \epsilon$.
                        So, any function $f$ with domain $\bbZ$ will be continuous at every point in $\bbZ$.
                    \end{proof}
                \item [(b)] Show in general that if $c$ is an isolated point of $A \subseteq \bbR$, then $f : A \to \bbR$ is continuous at $c$.
                    \begin{proof}
                        Assume $c$ is an isolated point of $A \subseteq \bbR$.
                        Need to show that $f : A \to \bbR$ is continuous at $c$, that is for all $V_\epsilon(f(c))$, there exists a $V_\delta(c)$ with the property that $x \in V_\delta(c)$ (and $x \in A$) implies $f(x) \in V_\epsilon(f(c))$.
                        Suppose $\epsilon > 0$ is arbitrary.
                        Since $c$ is an isolated point, we know from the inverse of Definition 3.2.4 that there exists a $\delta$ neighborhood $V_\delta(c)$ that intersects $A$ only at the point $c$, so $V_\delta(c) \cap A = c$.
                        Then, for all $x \in V_\delta(c)$, $x = c$ which means that $f(x) = f(c)$.
                        So, $f(x) \in V_\epsilon(f(c))$.
                        Thus, $f(x)$ is continuous at $c$ by Theorem 4.3.2 (iii).
                    \end{proof}
            \end{itemize}
        \item [4.3.6] 
            \begin{itemize}
                \item [(a)] Referring to the proper theorems, give a formal argument that Dirichlet’s function from Section 4.1 is nowhere-continuous on $\bbR$.
                    \begin{proof}
                        Let Dirichlet’s function be defined as follows:
                        \(
                            g(x) = 
                            \begin{cases}
                                1 \text{ if } x \in \bbQ \\
                                0 \text{ if } x \notin \bbQ.
                            \end{cases}
                        \)
                        Need to show that $g$ is nowhere-continuous -- that is it is not continuous at any point in $\bbR$.

                        First, choose $q\in\bbQ$ arbitrarily.
                        We know that $q \in \bbR$ and $\bbI$ is dense in $\bbR$ which means that there must exist a sequence $(x_n) \subseteq \bbI$ such that $(x_n) \to q$.
                        Then, for all $n \in \bbN$, $g(x_n) = 0$ which means $\lim g(x_n) = 0$, but $g(q) = 1$.
                        Since $\lim g(x_n) \neq g(q)$, we see that $g$ is not continuous at $q$ by Corollary 4.3.3 (Criterion for Discontinuity).

                        Next, choose $i \in \bbI$ arbitrarily.
                        We know that $i \in \bbR$ and $\bbQ$ is dense in $\bbR$ which means that there must exist a sequence $(y_n) \subseteq \bbQ$ such that $(y_n) \to i$.
                        Then, for all $n \in \bbN$, $g(y_n) = 1$ which means $\lim g(y_n) = 1$, but $g(i) = 0$.
                        Since $\lim g(y_n) \neq g(i)$, we see that $g$ is not continuous at $i$ by Corollary 4.3.3 (Criterion for Discontinuity).
                        
                        Since $\bbR$ consists of points from $\bbQ$ and $\bbI$, we see that $\bbR$ is nowhere-continuous.
                    \end{proof}
                \item [(b)] Review the definition of Thomae’s function in Section 4.1 and demonstrate that it fails to be continuous at every rational point.
                    \begin{proof}
                        Let Thomae’s function be defined as follows:
                        \[ t(x) =
                        \begin{cases}
                            1 & \text{if } x = 0\\
                            \frac{1}{n} & \text{if } x = \frac{m}{n} \in \bbQ - \{0\} \text{ is in lowest terms with } n > 0\\
                            0 & \text{if } x \notin \bbQ.
                        \end{cases}
                        \]
                        Arbitrarily choose $m,n \in \bbZ$ such that $n\neq 0$.
                        Need to show that $t$ is not continuous at $\frac{m}{n}$. 
                        Since $\frac{m}{n} \in \bbR$ and $\bbI$ is dense in $\bbR$, there must exist a sequence $(x_n) \subseteq \bbI$ such that $(x_n) \to \frac{m}{n}$.
                        Then, for all $n \in \bbN$, $t(x_n) = 0$ which means $\lim t(x_n) = 0$, but $t(\frac{m}{n}) = \frac{1}{n}$.
                        Since $\lim t(x_n) \neq t(\frac{m}{n})$, we see that $t$ is not continuous at $\frac{m}{n}$ by Corollary 4.3.3 (Criterion for Discontinuity).
                        Since $m,n$ were arbitrarily chosen and every $x \in \bbQ$ can be represented as $\frac{a}{b}$ for some $a,b \in \bbZ$ where $b \neq 0$, we see that $t$ is not continuous at any point in $\bbQ$.
                    \end{proof}
                \item [(c)] Use the characterization of continuity in Theorem 4.3.2 (iii) to show that Thomae’s function is continuous at every irrational point in $\bbR$. (Given $\epsilon > 0$, consider the set of points $\{x \in \bbR : t(x) \geq \epsilon\}$. Argue that all the points in this set are isolated.)
                    \begin{proof}
                        Choose $i \in \bbI$ arbitrarily.
                        Need to show that for all $V_\epsilon(t(i))$, there exists a $V_\delta(i)$ such that $v \in V_\delta(i)$ implies $t(v) \in V_\epsilon(t(i))$.
                        Choose $\epsilon > 0$ arbitrarily.
                        Let $T = \{x \in \bbR : t(x) \geq \epsilon\}$.
                        Then, for all $x\in T$, $x \in \bbQ$ so $\exists m,n \in \bbZ$ such that $x = \frac{m}{n}$ and $0 < n \leq \frac{1}{\epsilon}$.
                        Then, if we look at some fixed interval around $i$ like $[i-1, i+1]$, we can see that $T \cap [i-1, i+1]$ is finite since there are only a finite amount of rational numbers within this interval with a denominator less than $\frac{1}{\epsilon}$.
                        Since $T \cap [i-1, i+1]$ is finite, we can choose $\delta > 0$ such that for all $v \in V_\delta(i)$, $v \notin T$ which means that $t(v) \in V_\epsilon(t(i))$.
                        Thus, $t$ is continuous at every irrational point by Theorem 4.3.2 (iii).
                    \end{proof}
            \end{itemize}
        \newpage
        \item [4.3.8]
            \begin{itemize}
                \item [(a)] Show that if a function is continuous on all of $\bbR$ and equal to 0 at every rational point, then it must be identically 0 on all of $\bbR$.
                    \begin{proof}
                        Assume a function $f$ is continuous on all of $\bbR$ and equal to 0 at every rational point.
                        Since $f$ is 0 at every rational point, we only need to show that $f$ is 0 on every irrational point in $\bbR$.
                        Choose $i\in\bbI$ arbitrarily.
                        Since $\bbQ$ is dense in $\bbR$, we know that there exists a sequence $(x_n) \subseteq \bbQ$ such that $(x_n)\to i$.
                        Then, since $f$ is continuous on $\bbR$, we have that $0 = \lim f(x_n) = f(i)$.
                        So, $f$ is 0 on every irrational point in $\bbR$.
                        Thus, $f$ is 0 on all of $\bbR$.
                    \end{proof}
                \item [(b)] If $f$ and $g$ are defined on all of $\bbR$ and $f(r) = g(r)$ at every rational point, must $f$ and $g$ be the same function?
                    \begin{proof}
                        No, take the following counterexample:

                        \(
                            f(x) = 
                            \begin{cases}
                                0 \text{ if } x \in \bbQ \\
                                1 \text{ if } x \notin \bbQ
                            \end{cases}
                            \\
                            g(x) = 
                            \begin{cases}
                                0 \text{ if } x \in \bbQ \\
                                2 \text{ if } x \notin \bbQ
                            \end{cases}
                        \)

                        So, $f$ and $g$ are defined on all of $\bbR$ and $f(r) = g(r)$ at every rational point $r$.
                        However, $f$ and $g$ are not the same function.
                    \end{proof}
            \end{itemize}
        \item [4.3.9] (Contraction Mapping Theorem). Let $f$ be a function defined on all of $\bbR$, and assume there is a constant $c$ such that $0 < c < 1$ and $\abs{f(x) - f(y)} \leq c\abs{x-y}$ for all $x, y \in \bbR$.
            \begin{itemize}
                \item [(a)] Show that $f$ is continuous on $\bbR$.
                    \begin{proof}
                        Choose $\epsilon > 0$ and $y \in \bbR$ arbitrarily.
                        Need to show that $\exists \delta > 0$ such that $\abs{x - y} < \delta$ and $x \in \bbR$ implies that $\abs{f(x) - f(y)} < \epsilon$.
                        Choose $\delta = \frac{\epsilon}{c}$.
                        Then, $\abs{x - y} < \delta = \frac{\epsilon}{c}$ means that $c\abs{x-y} < \epsilon$.
                        So, $\abs{f(x) - f(y)} \leq c\abs{x-y} < \epsilon$.
                        Thus, $f$ is continuous on $\bbR$ by Theorem 4.3.1.
                    \end{proof}
                \item [(b)] Pick some point $y_1 \in \bbR$ and construct the sequence $(y_1, f(y_1), f(f(y_1)), ...)$. In general, if $y_{n+1} = f(y_n)$, show that the resulting sequence $(y_n)$ is a Cauchy sequence. Hence we may let $y = \lim y_n$.
                    \begin{proof}
                        Choose $y_1 \in \bbR$ arbitrarily.
                        Let $(y_n)$ be the sequence $(y_1, f(y_1), f(f(y_1)), ...)$.
                        Given an arbitrary $x \in \bbN$, we know from part a that $\abs{y_{x+1} - y_{x+2}} = \abs{f(y_{x}) - f(y_{x+1})} \leq c\abs{y_x - y_{x+1}} \leq c^2\abs{y_{x-1} - y_{x}} \leq ... \leq c^x\abs{y_{1} - y_{2}}$ where $0<c<1$.
                        Given arbitrary $i,j \in \bbN$ such that $i < j$, we know from the previous fact that $\abs{y_i - y_j} \leq \abs{y_i - y_{i+1}} + \abs{y_{i+1} - y_{i+2}} + ... + \abs{y_{j-1} - y_j} \leq c^{i-1}\abs{y_{1} - y_{2}} + c^i\abs{y_{1} - y_{2}} + ... + c^{j-2}\abs{y_{1} - y_{2}} = c^{i-1}\abs{y_{1} - y_{2}}(1 + c + ... + c^{j-i-1}) < c^{i-1}\abs{y_{1} - y_{2}}(\frac{1}{1-c})$.
                        Now, we need to show that $(y_n)$ is a cauchy sequence, that is for every $\epsilon > 0$ there exists an $N \in \bbN$ such that whenever $m,n \geq N$ with $m < n$ it follows that $\abs{y_m - y_n} = c^{m-1}\abs{y_{1} - y_{2}}(\frac{1}{1-c}) < \epsilon$.
                        Let $\epsilon > 0$ be arbitrary.
                        Choose $N \in \bbN$ such that $c^{N-1} < \frac{\epsilon(1-c)}{\abs{y_1 - y_2}}$.
                        Then, as shown above, whenever $m,n \geq N$ with $m < n$ it follows that $\abs{y_m - y_n} < \epsilon$ which satisfies that $(y_n)$ is a Cauchy Sequence.
                    \end{proof}
                \item [(c)] Prove that $y$ is a fixed point of $f$ (i.e., $f(y) = y$) and that it is unique in this regard.
                    \begin{proof}
                        Since $f$ is continuous, we know that $y = \lim y_n = lim y_{n+1} = \lim f(y_n) = f(y)$ as desired.
                    \end{proof}
                    \begin{proof}
                        Suppose $x$ is a fixed point so $x = f(x)$.
                        Then, $\abs{x-y} = \abs{f(x) - f(y)} \leq c\abs{x-y}$.
                        Since $0<c<1$, it must be that $x=y$ which means the fixed point is unique.
                    \end{proof}
                \item [(d)] Finally, prove that if $x$ is any arbitrary point in $\bbR$ then the sequence $(x, f(x), f(f(x)), ...)$ converges to $y$ defined in (b).
                    \begin{proof}
                        Assume $x$ is any arbitrary point in $\bbR$.
                        We know from part b that the sequence $(x, f(x), f(f(x)), ...)$ must converge to some limit $l$.
                        We also know from part c that $l$ must be a fixed point of $f$ and fixed points are unique so $l = y$.
                        Thus, the sequence $(x, f(x), f(f(x)), ...)$ converges to $y$ defined in (b).
                    \end{proof}
            \end{itemize}
        \item [4.3.11] For each of the following choices of $A$, construct a function $f : \bbR \to \bbR$ that has discontinuities at every point $x$ in $A$ and is continuous on $A^c$.
            \begin{itemize}
                \item [(a)] $A = \bbZ$.

                    The greatest integer function $f(x) = \lceil x \rceil$.
                \item [(b)] $A = \{x : 0 < x < 1\}$.

                \(
                    f(x) = 
                    \begin{cases}
                        0.5 - \abs{0.5-x} \text{ if } x \in \bbQ \text{ and } 0 < x < 1\\
                        0 \text{ if } x \notin \bbQ \text{ and } 0 < x < 1\\
                        0 \text{ otherwise }
                    \end{cases}
                \)
                \item [(c)] $A = \{x : 0 \leq x \leq 1\}$

                \(
                    f(x) = 
                    \begin{cases}
                        1 \text{ if } x \in \bbQ \text{ and } 0 \leq x \leq 1\\
                        0 \text{ if } x \notin \bbQ \text{ and } 0 \leq x \leq 1\\
                        0 \text{ otherwise }
                    \end{cases}
                \)
                \item [(d)] $A = \{\frac{1}{n} : n \in \bbN\}$.
                
                \(
                    f(x) = 
                    \begin{cases}
                        \frac{1}{n} \text{ if } x = \frac{1}{n} \text{ for some } n \in \bbN\\
                        0 \text{ otherwise }
                    \end{cases}
                \)
            \end{itemize}
        \newpage
        \item [4.4.3] Furnish the details (including an argument for Exercise 3.3.1 if it is not already done) for the proof of the Extreme Value Theorem (Theorem 4.4.3).
            \begin{proof}
                Assume $f: K\to\bbR$ is continuous on a compact set $K \subseteq \bbR$.
                Need to show that $f$ attains a maximum and a minimum value -- that is $\exists x_0, x_1 \in K$ such that $f(x_0) \leq f(x) \leq f(x_1)$ for all $x \in K$.
                Since $K$ is compact and $f:K \to \bbR$, we know that $f(K)$ is compact by Theorem 4.4.2 (Preservation of Compact Sets).
                Then, by Exercise 3.3.1 provided below, we know that since $f(K)$ is compact, it contains its supremum and infimum.
                Let $y_0 = \inf(f(K))$ and $y_1 = \sup(f(K))$.
                Then, since $y_0, y_1 \in f(K)$, there exists $x_0, x_1 \in K$ such that $f(x_0) = y_0$ and $f(x_1) = y_1$.
                By the definition of supremum and infiumum, we know that for all $x \in K, f(x_0) \leq f(x) \leq f(x_1)$.
            \end{proof}

                Exercise 3.3.1: If $K \subseteq \bbR$ is a compact set, then $\sup K$ and $\inf K$ both exist and are both elements of $K$.
                \begin{proof}
                    Assume $K \subseteq \bbR$ is a compact set.
                    By Theorem 3.3.4 (Heine-Borel Theorem), $K$ is closed and bounded.
                    So, we know from the Axiom of Completeness that $\sup K$ and $\inf K$ exist.
                    Let $s = \sup K$ and $i = \inf K$.
                    Choose any arbitray $\epsilon > 0$.
                    Then, we see that there exists $k \in K$ such that $k \in V_\epsilon(s)$, and since $k \in V_\epsilon(s)$ intersects $K$ in some point other than $s$, we know that $s$ must be a limit point of $K$ by Definition 3.2.4.
                    Similarly, we see that there exists $l \in K$ such that $l \in V_\epsilon(i)$, and since $l \in V_\epsilon(i)$ intersects $K$ in some point other than $i$, we know that $i$ must be a limit point of $K$ by Definition 3.2.4.
                    Since $s$ and $i$ are limit points of $K$ and $K$ is a closed set -- meaning $K = \overline{K}$ -- we know from Definition 3.2.11 that $s,i \in K$.
                \end{proof}
        \item [4.4.4] Show that if $f$ is continuous on $[a,b]$ with $f(x) > 0$ for all $a \leq x \leq b$, then $\frac{1}{f}$ is bounded on $[a,b]$.
                \begin{proof}
                    Suppose $a,b,x \in \bbR$ are arbitrarily chosen with $a \leq x \leq b$.
                    Assume $f$ is continuous on $[a,b]$ with $f(x) > 0$.
                    Need to prove that $\frac{1}{f}$ is bounded on $[a,b]$.
                    Since $[a,b]$ is closed and bounded, it is compact.
                    Because $f$ is continuous on the compact set $[a,b]$, we know from Theorem 4.4.3 (Extreme Value Theorem) that it has a minimum $m \in [a,b]$ such that $f(m) \leq f(k)$ for all $k \in [a,b]$.
                    Then, since $f(x) > 0$, this means $\frac{1}{f}$ is bounded below by $0$ and bounded above by $\frac{1}{f(m)}$.
                    Thus, $\frac{1}{f}$ is bounded on $[a,b]$.
                \end{proof}
        \newpage
        \item [4.4.6] Give an example of each of the following, or state that such a request is impossible. For any that are impossible, supply a short explanation (perhaps referencing the appropriate theorem(s)) for why this is the case.
            \begin{itemize}
                \item [(a)] a continuous function $f : (0,1) \to \bbR$ and a Cauchy sequence $(x_n)$ such that $f(x_n)$ is not a Cauchy sequence;
                
                    Let $f(x) = \frac{1}{x}$ and $x_n = \frac{1}{n}$. Then, we see that $(x_n) \to 0$ so it is cauchy. But, $f(x_n) = n$ and that does not converge meaning it is not a cauchy sequence.
                \item [(b)] a continuous function $f : [0,1] \to \bbR$ and a Cauchy sequence $(x_n)$ such that $f(x_n)$ is not a Cauchy sequence;
                
                    Impossible, because $f$ is continuous and $[0,1]$ is compact which means that any Cauchy sequence $(x_n) \in [0,1]$ converges to a limit point $\lim (x_n) \in [0,1]$. 
                    Then, we know by continuity that $f(x_n)$ must converge to $f(\lim (x_n))$ and so it is a Cauchy sequence.
                \item [(c)] a continuous function $f : [0,\infty) \to \bbR$ and a Cauchy sequence $(x_n)$ such that $f(x_n)$ is not a Cauchy sequence;
                
                    Impossible, because $f$ is continuous and $[0, \infty)$ contains all its limit points and so any cauchy sequence $(x_n) \in [0,\infty)$ converges to a limit point $\lim (x_n) \in [0, \infty)$.
                    Then, we know by continuity that $f(x_n)$ must converge to $f(\lim (x_n))$ and so it is a Cauchy sequence.
                \item [(d)] a continuous bounded function $f$ on $(0,1)$ that attains a maximum value on this open interval but not a minimum value.

                    $f(x) = 0.5 - \abs{0.5-x}$. The maximum value is $0.5$. There is no minimum as desired.
            \end{itemize}
        \item [4.4.9] A function $f : A \to \bbR$ is called Lipschitz if there exists a bound $M > 0$ such that $\abs{\frac{f(x) - f(y)}{x-y}} \leq M$ for all $x,y \in A$. Geometrically speaking, a function $f$ is Lipschitz if there is a uniform bound on the magnitude of the slopes of lines drawn through any two points on the graph of $f$.
            \begin{itemize}
                \item [(a)] Show that if $f : A \to \bbR$ is Lipschitz, then it is uniformly continuous on $A$.
                    \begin{proof}
                        Assume that $f : A \to \bbR$ is Lipschitz.
                        Need to show that $f$ is uniformly continuous on $A$, that is $\forall \epsilon > 0$, $\exists \delta > 0$ such that $\abs{x-y} < \delta$ implies $\abs{f(x) - f(y)} < \epsilon$ where $x, y \in A$.
                        Suppose $\epsilon > 0$ and $x,y \in A$ are arbitrary.
                        Since $f$ is Lipschitz, there exists a bound $M > 0$ such that $\abs{\frac{f(x) - f(y)}{x-y}} \leq M$.
                        Then, choose $\delta = \frac{\epsilon}{M}$.
                        So, $\abs{\frac{f(x) - f(y)}{x-y}} \leq M$ becomes $\abs{f(x) - f(y)} \leq M \abs{x - y} < M \frac{\epsilon}{M} = \epsilon$.
                        Thus, $f$ is uniformly continuous on $A$.
                    \end{proof}
                \item [(b)] Is the converse statement true? Are all uniformly continuous functions necessarily Lipschitz?
                    
                    No.
                    As a counterexample take $f(x) = \sqrt(x)$ on $[0, 1]$.
                    We see that $[0,1]$ is compact and $f$ is continuous on $[0,1]$ which makes it uniformly continuous on $[0,1]$.
                    However, it is not Lipschitz because we see that as $x \to 0$, the magnitude of the slope of $f(x)$ approaches $\infty$ -- this means that the magnitude of the slope of $f(x)$ is unbounded -- and so it fails to satisfy the geometric explanation of Lipschitz provided in the problem description.
            \end{itemize}
        \item [4.4.10] Do uniformly continuous functions preserve boundedness? If $f$ is uniformly continuous on a bounded set $A$, is $f(A)$ necessarily bounded?
        
            Yes
            \begin{proof}
                Assume $f$ is uniformly continuous on a bounded set $A$.
                Need to show that $f(A)$ is bounded.
                Let $\epsilon > 0$ be arbitrary.
                Since $f$ is uniformly continuous, there exists a $\delta > 0$ such that $\abs{x-y} < \delta$ implies $\abs{f(x) - f(y)} < \epsilon$ for $x,y \in A$.
                Since $A$ is bounded, there are a finite set of points $\{x_1, ..., x_n\}$ such that $\bigcup_{1 \leq i \leq n}(x_i - \delta, x_i + \delta)$ is a finite open cover of $A$.
                This means $\{f(x_1), ..., f(x_n)\}$ is also finite. 
                Thus, since it's finite $f(A)$ is necessarily bounded as desired.
            \end{proof}
        \item [4.4.11] (Topological Characterization of Continuity). Let $g$ be defined on all of $\bbR$. If $A$ is a subset of $\bbR$, define the set $g^{-1}(A)$ by $g^{-1}(A) = \{x \in \bbR : g(x) \in A\}$. Show that $g$ is continuous if and only if $g^{-1}(O)$ is open whenever $O \subseteq \bbR$ is an open set.
            \begin{proof}
                First we show that if $g$ is continuous then $g^{-1}(O)$ is open whenever $O \subseteq \bbR$ is an open set.
                Assume $g$ is continuous.
                Suppose $O \subseteq \bbR$ is an open set.
                Need to show that $g^{-1}(O)$ is open, that is for all points $a \in g^{-1}(O)$ there exists a $\delta$-neighborhood $V_\delta(a) \subseteq g^{-1}(O)$.
                Choose an arbitrary $a \in g^{-1}(O)$.
                Then, we know from the definition of $g^{-1}$ that $g(a) \in O$.
                Since $O$ is open, we know that for all points $x \in O$ there exists a $\epsilon$-neighborhood $V_\epsilon(x) \subseteq O$.
                Choose an $\epsilon > 0$ such that $V_\epsilon(g(a)) \subseteq O$.
                Since $g$ is continuous, this implies that there exists a $\delta$-neighborhood $V_\delta(a)$ such that for all $x \in V_\delta(a)$, $g(x) \in V_\epsilon(g(a))$.
                This means $V_\delta(a) \subseteq g^{-1}(O)$.
                So, $g^{-1}(O)$ is open.

                Next, we need to show that if $g^{-1}(O)$ is open whenever $O \subseteq \bbR$ is an open set, then $g$ is continuous -- that is given an arbitrary point $c \in \bbR$, for all $V_\epsilon(g(c))$ there exists a $V_\delta(c)$ such that $x \in V_\delta(c)$ implies $g(x) \in V_\epsilon(g(c))$.
                Choose an arbitrary point $c \in \bbR$ and $\epsilon > 0$. 
                Let $O$ be the open set given by $V_\epsilon(g(c))$.
                This means that $g^{-1}(O)$ is open -- that is, there exists a $\delta > 0$ such that $V_\delta(c) \subseteq g^{-1}(O)$.
                Choose such a $\delta$.
                Then, for all $x \in V_\delta(c)$, $g(x) \in O$ which means $g(x) \in V_\epsilon(g(c))$ as desired.
                Thus, $g$ is continuous by Theorem 4.3.2 (iii) (Characterizations of Continuity).
            \end{proof}
        \newpage
        \item [4.5.2] Decide on the validity of the following conjectures.
            \begin{itemize}
                \item [(a)] Continuous functions take bounded open intervals to bounded open intervals.
                
                    False. A counterexample would be the continuous function $f(x) = \frac{1}{x}$ which takes the bounded open interval $(0,1)$ to the unbounded open interval $(1, \infty)$.
                \item [(b)] Continuous functions take bounded open intervals to open sets.

                    False. A counterexample would be the continuous function $f(x) = 1-x^2$ which takes the bounded open interval $(-1,1)$ to the set $(0, 1]$ which is not an open set.
                \item [(c)] Continuous functions take bounded closed intervals to bounded closed intervals.
                
                    True. A bounded and closed interval on $\bbR$ will behave identically to a compact set $K \subseteq \bbR$ and we know that a continuous function on a compact set maps to another compact set by Theorem 4.4.2 (Preservation of Compact Sets).
            \end{itemize}
        \item [4.5.3] Is there a continuous function on all of $\bbR$ with range $f(\bbR)$ equal to $\bbQ$?
        
            No, there isn't because it would fail to contain the irrational points since $\bbQ$ is not connected. A specific example would be to assume that 1 and 2 were in the range of $f$, then by IVT, $f$ would also have to contain $\sqrt(2)$, but $\sqrt(2) \notin \bbQ$.
        \item [4.5.6] Finish the proof of the Intermediate Value Theorem using the Nested Interval Property started previously.
        
            \textbf{Textbook}: 
                Consider the special case where $L = 0$ and $f(a) < 0 < f(b)$ where $a,b \in \bbR$.
                Let $I_0 = [a,b]$, and consider the midpoint $z = \frac{a+b}{2}$.
                If $f(z) \geq 0$, then set $a_1 = a$ and $b_1 = z$. 
                If $f(z) < 0$, then set $a_1 = z$ and $b_1 = b$.
                In either case the interval $I_1 = [a_1, b_1]$ has the property that $f$ is negative at the left endpoint and non-negative at the right.
            
            \ \\
            \textbf{Continuation of the proof:} 
            \begin{proof}
                We can set $z$ to be the midpoint of the new interval and inductively continue the construction so we see that $\forall n \in \bbN$ the interval $I_n = [a_n, b_n]$ where the left endpoint is negative [so $f(a_n) < 0$] and the right endpoint is non-negative [so $f(b_n) \geq 0$].
                Then, by the Nested Interval Property, we know that there exists an $x \in \bigcap_{n=1}^{\infty}I_n$ which is the point that both $(a_n)$ and $(b_n)$ are converging towards.
                Since $f$ is continuous, we know that $\lim f(a_n) = f(x) = \lim f(b_n)$.
                Then we know from the Order Limit Theroem that since for all $n \in \bbN$ $f(a_n) < 0$ and $f(b_n) \geq 0$, it must be true that $f(x) \leq 0$ and $f(x) \geq 0$. 
                This means that $f(x) = 0$ as desired.
            \end{proof}
    \end{itemize}
\end{document}